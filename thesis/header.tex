    \usepackage{makecell}
    \usepackage{amsmath}
    \usepackage{setspace}
    \usepackage{amssymb}
    \usepackage{amsthm}
    \usepackage{thmtools}
    \usepackage{mathtools}
    \usepackage{listings}
    \usepackage[utf8]{inputenc}
    \usepackage[]{appendix}
    \usepackage{float}
    \usepackage{xcolor}

    % Landscape fullpage tables
    \usepackage{pdflscape}
    \usepackage{afterpage}
    \usepackage{geometry}

    % Nested file structure of thesis
    \usepackage{import}

    % Drawing graphs
    % \usepackage{tikz}
    \usepackage{blindtext}
    \newcommand\void[1]{}

    % Plotting data
    \usepackage{pgfplots}
    \usepackage{pgfplotstable}
    \pgfplotsset{filter discard warning=false}

% \scatterPlot{x values}{y values}{file}
\newcommand{\scatterPlot}[4][]{%
    \addplot+[only marks, mark=x, #1] table [#2, #3] {#4};
}

% \linRegression{x values}{y values}{file}
\newcommand{\linRegression}[4][]{%
    \addplot+[no markers, red, dashed, axis on top, #1] table [#2, y={create col/linear regression={#3}}] {#4};
}

% \histogram{values}{file}
\newcommand{\histogram}[3][]{
    \addplot+[no markers, hist={bins=40}, fill=blue!20, draw=blue, #1] table [#2] {#3};
}

\pgfmathdeclarefunction{ghelper}{2}{%
\pgfmathparse{((1+1/#2)^#1)/(1+#1/#2)}%
}
\pgfmathdeclarefunction{gamma}{1}{%
\pgfmathparse{1/#1*ghelper(#1,1)*ghelper(#1,2)*ghelper(#1,3)*ghelper(#1,4)*ghelper(#1,5)*ghelper(#1,6)*ghelper(#1,7)*ghelper(#1,8)*ghelper(#1,9)*ghelper(#1,10)}%
}
\pgfmathdeclarefunction{gammaPDF}{2}{%
\pgfmathparse{1/(#2^#1*gamma(#1))*x^(#1-1)*exp(-x/#2)}%
}

\newcommand{\gammaDistribution}[3]{
\addplot[
    red, dashed,
    mark=none,
    samples=20,
    smooth,
    domain=1:10,
] {gammaPDF(#1,#2)*#3};
}

\newcommand{\scatterSplit}[4]{
    \scatterPlot[y filter/.expression={#1 ? y : nan}]{#2}{#3}{#4}
    \scatterPlot[black, y filter/.expression={#1 ? nan : y}]{#2}{#3}{#4}
}

\newcommand{\plotDrawLine}[3][]{
    \draw [red, dashed, line width=1.5pt, #1] (#2) -- (#3);
}

\newcommand{\drawHLine}[2][]{
    \plotDrawLine[#1]{axis cs:\pgfkeysvalueof{/pgfplots/xmin},#2}{axis cs:\pgfkeysvalueof{/pgfplots/xmax},#2};
}
\newcommand{\drawVLine}[2][]{
    \plotDrawLine[#1]{axis cs:#2,\pgfkeysvalueof{/pgfplots/ymin}}{axis cs:#2,\pgfkeysvalueof{/pgfplots/ymax}};
}

    \usetikzlibrary{math}
    \tikzmath{
    function choice(\n, \k) {
        return factorial(\n) / (factorial(\n - \k) * factorial(\k));
    };
    function hamming(\s, \d) {
        \k = 0;
        for \i in {0,1,2,3,4,5,6,7,8}{
            if \d >= \i then {
                \k = \k + choice(\s, \i);
            };
        };
        return \k;
    };
    function fixedbits(\f) {
        return 2^\f;
    };
}

    \usepackage{xstring}

    % Clever references
    \usepackage{hyperref}
    \usepackage[nameinlink,noabbrev,capitalize,sort]{cleveref}

    \Crefname{appsec}{Appendix}{Appendices}
    \AtBeginEnvironment{appendices}{\crefalias{chapter}{appsec}}

    \newfloat{Encoding}{htbp}{loen}
    %\crefalias{enc}{encoding}
    \newenvironment{code}
    {
        \setlength{\abovecaptionskip}{0.5em}
        \let\outerlabel\label
        \renewcommand{\label}[1]{\outerlabel[encoding]{##1}}
    }{
        \vspace{0.5em}
    }
    \Crefname{encoding}{Encoding}{Encodings}
    \AtBeginEnvironment{encoding}{\crefalias{table}{encoding}}

    % BNN representation
    \newcommand{\BB}{\mathbb{B}}       % {-1, 1}
    \newcommand{\RR}{\mathbb{R}}       % real numbers
    \newcommand{\QQ}{\mathbb{Q}}       % rational numbers
    \newcommand{\NN}{\mathbb{N}}       % natural numbers
    \newcommand{\ZZ}{\mathbb{Z}}       % whole numbers
    \newcommand{\BNN}{\mathcal{N}^{\BB}}     % binary neural network
    \newcommand{\DNN}{\mathcal{N}}     % binary neural network
    \newcommand{\mat}[1]{\mathbf{#1}}  % matrix
    \newcommand{\sgn}{\mathrm{sgn}}  % signum function

    % Custom operators
    \newcommand{\floor}[1]{\left\lfloor{} #1 \right\rfloor{}}
    \DeclareMathOperator*{\argmax}{arg\,max}
    \DeclareMathOperator*{\argmin}{arg\,min}
    \DeclareMathOperator*{\argsort}{arg\,sort}

    % ASP representation
    \newcommand{\asprule}[2]{\ensuremath{#1 \leftarrow{} #2.}}  % head <- body.
    \newcommand{\aspfact}[1]{\ensuremath{#1.}}  % head.
    \newcommand{\aspconstraint}[1]{\ensuremath{\leftarrow{} #1.}}
    \newcommand{\aspnot}{\mathrm{not}\,}          % not p
    \newcommand{\asphead}{\mathrm{head}}          % head(r)
    \newcommand{\aspbody}{\mathrm{body}}          % body(r)
    \newcommand{\aspCn}[1]{\mathrm{Cn}(#1)}              % Cn(\Pi)

    % Tables
    \newcolumntype{L}{>{$}l<{$}}  % math column
    \newcolumntype{C}{>{$}c<{$}}  % math column
    \newcolumntype{R}{>{$}r<{$}}  % math column
    \setlength{\tabcolsep}{1em}   % bigger column sep for readability

    % Listings
    \lstdefinestyle{mystyle}{
    %    backgroundcolor=\color{backcolour},
    %    commentstyle=\color{codegreen},
    %    keywordstyle=\color{magenta},
    %    numberstyle=\tiny\color{codegray},
    %    stringstyle=\color{codepurple},
        basicstyle=\ttfamily\footnotesize,
        breakatwhitespace=false,
        breaklines=true,
        captionpos=b,
        keepspaces=true,
        numbers=left,
        numbersep=5pt,
        showspaces=false,
        showstringspaces=false,
        showtabs=false,
        tabsize=2
    }
    \lstset{style=mystyle}
    \newcommand{\lstclingo}[1]{\lstinputlisting[language=prolog]{programs/#1}}

    \lstset{numbers=left,numberblanklines=false,escapeinside=||}
    \let\origthelstnumber\thelstnumber{}

    \makeatletter
    \lst@Key{countblanklines}{true}[t]%
        {\lstKV@SetIf{#1}\lst@ifcountblanklines}

    \lst@AddToHook{OnEmptyLine}{%
        \lst@ifnumberblanklines\else%
           \lst@ifcountblanklines\else%
             \advance\c@lstnumber-\@ne\relax%
           \fi%
        \fi}

    \newcommand*\Suppressnumber{%
      \lst@AddToHook{OnNewLine}{%
        \let\thelstnumber\relax%
         \advance\c@lstnumber-\@ne\relax%
        }%
    }

    \newcommand*\Reactivatenumber{%
      \lst@AddToHook{OnNewLine}{%
       \let\thelstnumber\origthelstnumber%
       \advance\c@lstnumber\@ne\relax}%
    }
    \makeatother

    % Writting rules etc
    \newcommand\fact[2]{\texttt{#1(}#2\texttt{).}\\}  % chktex 9
    \newcommand\atom[2]{\texttt{#1($#2$)}}


    % Evaluation
    \newcounter{hypothesis}
    \newcommand\hypothesis[1]{%
        \stepcounter{hypothesis}%
        \noindent\textbf{Hypothesis \thehypothesis}:\\#1%
        }

    \newcommand{\landtable}[1]{%
        \afterpage{%
            \clearpage%
            \newgeometry{margin=0in}%
            \thispagestyle{empty}%
            \begin{landscape}%
                \begin{table}%
                    \centering%
                    #1% tabular, captions, label
                \end{table}%
            \end{landscape}%
            \clearpage%
            \restoregeometry%
        }%
    }%

    \newcommand{\sectionsep}{%
        \vspace{1em}\hrule{}\vspace{1em}%
    }


    % Theorems, definitions, etc.
    \theoremstyle{plain}
    \crefname{theorem}{Theorem}{Theorems}
    \newtheorem{theorem}{Theorem}[section]
    \crefname{lemma}{Lemma}{Lemmas}
    \newtheorem{lemma}[theorem]{Lemma}

\theoremstyle{definition}
\crefname{definition}{Definition}{Definitions}
\newtheorem{definition}{Definition}[section]
\crefname{example}{Example}{Examples}
\newtheorem{example}{Example}[section]

\theoremstyle{remark}
\crefname{remark}{Remark}{Remarks}
\newtheorem*{remark}{Remark}

\Crefname{proof}{Proof}{Proofs}
\AtBeginEnvironment{appendix}{\crefalias{lemma}{proof}}


% Import of chapters
\newcommand{\importchapter}[1]{%
    \import{chapters/#1/}{chapter.tex}%
}

% Special characters
\DeclareUnicodeCharacter{21B5}{$\hookleftarrow$}

% Drawing perceptron networks
% chktex-file 1
% chktex-file 8
\usetikzlibrary{shapes.geometric}
\usetikzlibrary{shapes.multipart}
\usetikzlibrary{calc}

\newlength{\nodesize}
\setlength{\nodesize}{1.5em}

\tikzset{inp/.style = {shape=rectangle,draw,left}}
\tikzset{perc/.style = {rectangle split, rectangle split parts=2, rectangle split horizontal,rounded corners, draw, left}}
\newcommand{\setupnodes}[1]{
    \node (lastNode) at #1 {};
}
\newcommand\nsep{(0,-1)}
\newcommand{\valuenode}[1]{
    \node[inp] (#1) at (lastNode) {$#1$};
    \node (lastNode) at ($(#1.south east) + (0,-1)$) {};
}
\newcommand{\percnode}[3][]{
    \foreach \l/\p in {#2}{
        \node[perc] (p{#2}) at (lastNode) { $\xi_{\p}^{\l}$ \nodepart{two} $\sigma_{\p}^{\l}$ };
    }
    \node (lastNode) at ($(p{#2}.south east) + (0,-1)$) {};
    \foreach \x/\w in {#3}{\draw[#1] (\x.east) -- (p{#2}.west) node[near start,above] {$\w$};}
}

\tikzset{const/.style = {shape=rectangle,draw}}
\tikzset{comp/.style = {rectangle,rounded corners,draw}}
\newcommand{\constnode}[1]{
    \foreach \sym/\arg in {#1}{
        \node[const] (const{#1}) at (lastNode) {\atom{\sym}{\arg}};
    }
    \node (lastNode) at ($(const{#1}.south) + (0,-1)$) {};
}
\newcommand{\compnode}[4][]{
    \foreach \sym/\arg in {#2}{
        \node[comp] (comp{#2}) at (lastNode) {\atom{\sym}{\arg}};
    }
    \node (lastNode) at ($(comp{#2}.south) + (0,-1)$) {};
    \foreach \x in {#3}{\draw[#1] (const{\x}.east) -- (comp{#2}.west);}
    \foreach \x in {#4}{\draw[#1] (comp{\x}.east) -- (comp{#2}.west);}
}
