\chapter*{Introduction}
\addcontentsline{toc}{chapter}{Introduction}

Deep neural networks (DNN) are state-of-the-art technology. They are used in many
real-world applications e.g.\ medicine, self-driving cars, autonomous systems,
many of which are critical.
Deep neural networks used in natural language processing have up to hundreds of
billions of parameters~\cite{2021arXiv210901652W}.
That is too many for people to comprehend. Tools for the automation
of DNN verification are needed.

%For the qualitative disproving of the verity of general DNNs,
%many algorithms for finding adversarial
%inputs have already been developed~\cite{7958570, chen2020real, 8578273}.
%The DNN can be qualitatively disproven by finding any adversarial input.
%On the other hand, proving the nonexistence of adversarial input is usually made by generating constraints
%on adversarial inputs and then showing there can be no such
%input~\cite{Isac2022NeuralNV}. This is computationally much more difficult
%and prone to errors emerging from inaccuracies of floating point numbers operations.
%It is often made with the use of a simplex algorithm combined with 
%the satisfiability modulo theories (SMT) paradigm~\cite{Isac2022NeuralNV, marabou2019, reluplex2017}.
%As far as I am aware, there is no reasonably quick quantitative
%validator of general DNNs yet.

While DNNs generally base their computation on floating point numbers,
another branch of DNN development, quantized neural networks
use for their computation low-bit-width values.
This can lower both the size of neural network and power consumption
while increasing speed of tnhe network~\cite{jacob2017,han2016}.
An extreme example of quantization are Binarised neural networks (BNN)~\cite{nips2016},
which use binary parameters for their computations, resulting in much
cheaper computational price, both when learning and in production,
while maintaining near state-of-the-art results~\cite{Agrawal2023}.

Verification of neural networks is split into two categories. First,
qualitative verification, searches through input space looking for any
adversarial input, that is an input which results in a wrong output.
Second, quantitative verification, searches through input space
determining the portion of the input space giving adversarial outputs.
While both qualitative and quantitative verificators for BNNs do exist,
they scale only up to millions of parameters for qualitative verification~\cite{10.1007/978-3-030-03592-1_16}
and tens of thousands of parameters for quantitative verification~\cite{10.1145/3563212}.
\vspace{1em}

In this thesis, I provide definition of binarised neural network and the problems
of both qualitative and quantitative robustness over arbitrary function.
Further I provide a definition of an input region based on hamming distance from
base input and based on fixed bits.

The third chapter focusses on Answer set programming (ASP)~\cite{lifschitz2008answer,anger2005glimpse}.
In it I provide a definition of a logic program and semantics of ASP using
the reduct of a program~\cite{gelfond1991}.
I also show the syntax and semantics of Clingo~\cite{gebser2019potassco},
an integrated ASP solver.

I then implement a framework for quantitative robustness verification of BNNs
using Clingo. This implementation allows for verification of input regions
based on hamming distance and fixed bits as in~\cite{zhang2021bdd4bnn}.
As far as I am concerned, this is the first such implementation using
the ASP paradigm.

Finally, I evaluate the speed of this verificator on BNN models of various
architectures based of PyTorch deep learning platform provided by NPAQ~\cite{baluta2019quantitative}
trained in~\cite{10.1145/3563212} on MNIST~\cite{mnist2017} dataset.
