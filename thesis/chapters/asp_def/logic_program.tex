\section{Extended logic program}

Extended logic program is a finite set of facts, rules and constraints
each consisting literals. A literal is an atom or its negation.
An atom is the elementary construct for representing knowledge~\cite{Delgrande}.
Each atom constitutes a single variable, it can be seen as a possible feature of solution.

In order to be able to distinguish between a query which fails in the sense
that it does not succeed and a query which fails in the stronger sense
that its negation succeeds, extended logic programs allow
for the \textit{classical negation~$\neg$}
in addition to the \textit{negation-as-failure~not}~\cite{gelfond1991}.

Further I provide formal definition of extended logic programs according to~\cite{gelfond1991,KRHandbook}.
I will assume the notation of signature as defined in~\cite{zarba2006}.

% \begin{definition}{Signature.}
%     A (many-sorted) signature is a tuple $\Sigma = (S, C, F, P)$, where:
%     \begin{itemize}
%         \item $S$ is a nonempty set of sorts, each sort $\sigma\in S$
%             is associated with a set of variables $\mathbf{Var}_\sigma$ and
%             set of constants $\mathbf{Con}_\sigma$,
%         \item $C$ is a countable set of constant symbols whose sorts belong to~$S$,
%         \item $F$ is a countable set of function symbols whose arities
%             are constructed using sorts that belong to~$S$,
%         \item $P$ is a countable set of predicate symbols whose arities
%             are constructed using sorts that belong to~$S$.
%     \end{itemize}
%     We denote with $\mathbf{Fun}_{\sigma_1\times\cdots\times\sigma_n\to \sigma}$
%     the set of function symbols of arity $\sigma_1\times\cdots\times\sigma_n\to \sigma$.
%     We denote with $\mathbf{Pred}_{\sigma_1\times\cdots\times\sigma_n\to \sigma}$
%     the set of function symbols of arity $\sigma_1\times\cdots\times\sigma_n\to \sigma$.
%     Given a signature $\Sigma = (S, C, F, P)$, we write $\Sigma^{S}$ for $S$,
%     $\Sigma^{C}$ for $C$, $\Sigma^{F}$ for $F$ and $\Sigma^{P}$ for $P$~\cite{zarba2006}.
% \end{definition}

\subsection{Syntax of Extended logic program}

\begin{definition}{Terms~\cite{zarba2006}.}
    Let $\Sigma$ be a signature. The set of $\Sigma$-terms of sort $\sigma$
    is the smallest set of expressions satisfying the following properties:
    \begin{itemize}
        \item Each variable $x$ of sort $\sigma$ is a term of sort $\sigma$, provided that $\sigma\in \Sigma^S$.
        \item Each constant symbol $c\in \Sigma^C$ of sort $\sigma$ is a $\Sigma$-term of sort $\sigma$.
        \item If $f\in \Sigma^F$ is a function symbol of arity $\sigma_1\times\cdots\times \sigma_n\to \sigma$
            and $t_i$ is a $\Sigma$ term of $\sigma_i$, for $i=1,\ldots, n$, then $f(t_1,\ldots,t_n)$ is a term of sort $\sigma$.
    \end{itemize}
\end{definition}

\begin{definition}{Atoms, literals~\cite{zarba2006}.}
    Let $\Sigma$ be a signature.
    A $\Sigma$-atom is an expression of the form
    \begin{equation*}
        p(t_1, \ldots, t_n)
    \end{equation*}
    where $p\in \Sigma^P$ is a predicate symbol of arity $\sigma_1\times\cdots\times\sigma_n$
    and for $i=1,\ldots,n$, $t_i$ is a $\Sigma$-term of sort $\sigma_i$.

    A $\Sigma$-literal is a formula of the form
    \begin{equation*}
        \varphi\, \text{ or } \,\neg\varphi
    \end{equation*}
    where $\varphi$ is a $\Sigma$-atom.
\end{definition}

For both function and predicate symbols infix notation can be used if was estabilished
(e.g.\ $+$, $-$, $=$, $\leq$, \ldots). If this notation would result in ambiguity, brackets should be used.

\begin{example}%
    \label{Sigma1}%
    Let $\Sigma_1$ be a signature with sorts
    \begin{equation*}
        \sigma_\BB = \{True, False\}, \sigma_\NN = \NN_0
    \end{equation*}
    and with function symbols $\vee, \wedge$ of arity $\sigma_\BB\times \sigma_\BB\to \sigma_\BB$,
    and function symbols $=, <$ of arity $\sigma_\NN\times\sigma_\NN\to \sigma_\BB$.
    Let $p$ be a predicate of arity $\sigma_\BB$, and $q$ be a predicate of arity $\sigma_\NN\times\sigma_\BB$
    Also let variables $X, Y, Z$ be of sort $\sigma_\NN$ and let there be no variables of sort $\sigma_\BB$.\\
    Then expressions
    \begin{gather*}
        True,\ False,\\
        True \vee False,\ False \wedge True,\\
        (42 = 42),\ (X = 17),\ (3 < Y) \vee (6 = 6),\\
        [(5 = 1)\wedge True] \vee False
    \end{gather*}
    are some of the $\Sigma$-terms of sort $\sigma_\BB$.\\
    Expressions
    \begin{gather*}
        p(True),\ p(False),\\
        p([(5 = 1) \wedge True] \vee False),\\
        q(0, True),\ q(42, False),\\
        q(123, (42 = 42)),\ q(93, True \vee (Z < 17))
    \end{gather*}
    are some of the $\Sigma$-atoms.
\end{example}

\begin{definition}{Rule~\cite{KRHandbook}.}\label{def:rule}
    Let $\Sigma$ be a signature. A $\Sigma$-rule ($\Sigma$-formula) $r$ is an expression of the form
    \begin{equation*}
        \asprule{l_0\, or\, \ldots\, or\, l_k}{l_{k+1}, \ldots, l_{m}, \aspnot l_{m+1}, \ldots, \aspnot l_{n}}
    \end{equation*}
    where each $l_i$ is a $\Sigma$-literal. The following notation is used:
    \begin{equation*}
        \asphead(r) = \{l_0, \ldots, l_k\}
    \end{equation*}
    \begin{equation*}
        \aspbody(r) = \{l_{k+1}, \ldots, l_{m}, \aspnot l_{m+1}, \ldots, \aspnot l_n\}
    \end{equation*}
    \begin{equation*}
        \aspbody^+(r) = \{l_{k+1}, \ldots, l_{m}\}
    \end{equation*}
    \begin{equation*}
        \aspbody^-(r) = \{\aspnot l_{m+1}, \ldots, \aspnot l_n\}
    \end{equation*}
    Further, if $\asphead(r) = \emptyset$, the rule is called a constraint and is written as
    \begin{equation*}
        \asprule{}{l_{k+1}, \ldots, l_{m}, \aspnot l_{m+1}, \ldots, \aspnot l_{n}}
    \end{equation*}
    If $\aspbody(r) = \emptyset$, the rule is called a fact and is written as
    \begin{equation*}
        l_0\, or\, \ldots\, or\, l_k.
    \end{equation*}

    The term $\aspnot$ is called default negation.
\end{definition}

\begin{definition}{Logic program~\cite{KRHandbook}.}
    Logic program is a pair $(\Sigma, \Pi)$ where $\Sigma$ is a signature
    and $\Pi$ is a collection (set) of $\Sigma$-rules.
\end{definition}

In this thesis, I will be mostly working with logic programs that have
zero or one literal in the head of each of their rules.
While allowing for multiple literals in the head of a rule can allow
for otherwise impossible to achieve expressions, it does introduce
a large portion of nontriviality and even computational cost.
General logic programs are in the complexity class $\Sigma_2^P$~\cite{dis_datalog_complexity},
logic programs without disjunction in head are $NP$-complete~\cite{schlipf1995_computation}
and logic programs with neither disjunction nor default negation belong to $P$~\cite{KRHandbook}.

\begin{example}\label{exp:logic_program1}
    An example of a logic program with signature $\Sigma_1$ from Example~\ref{Sigma1}
    is a pair $(\Sigma_1, \Pi_1)$ where $\Pi_1$ is a set
    \begin{equation*}
        \Pi_1 =
            \begin{cases}
                \aspfact{p(True)}\\
                \aspfact{q(5, True)}\\
                \asprule{p(False)}{\aspnot p(True)}\\
                \asprule{q(Y, True)}{q(X, (Y < X))}
            \end{cases}
    \end{equation*}
\end{example}

The logic program is often denoted only by its second element $\Pi$.
In that case its signature consists of symbols occuring in the program.

% Poznámka: různé programy pro řešení umožňují použití rozšířené definice
% pro tuto práci si však vystačíme s touto a více nebudeme uvažovat.

\subsection{Semantics of extended logic program}

For the definition of the extended logic programs to be usefull,
definition of stable models is needed.
Many definitions have been created~\cite{12definitions}.
Here I will show the definition using the reduct
of a logic program~\cite{KRHandbook},
which was used for the ASP solver I am using for the implementation
of verification, Clingo~\cite{GebserKKS17}.

First of all, the logic program needs to be grounded.
Grounding is a process in which we substitute each rule
in the program for its equivalent rules with constant symbols only.

The semantics of function symbols is usually defined in the groun\-d\-er
(the program that does the grounding)
thus the programs can be grounded before the start of solving
of the logic program.

\begin{definition}{Ground program~\cite{KRHandbook}.}
    Terms, literals, and rules of program $\Pi$ with signature $\sigma$ are called ground if they
    contain no variables and no function symbols. A program is called
    ground if all its rules are ground. A rule $r'$ is called a ground instance of a rule $r$ of $\Pi$
    if it is obtained from $r$ by:
    \begin{itemize}
        \item replacing $r'$s variables by properly typed ground terms of $\sigma$;
        \item replacing $r'$s function terms by their values.
    \end{itemize}
    A program $gr(\Pi)$ consisting of all ground instances of all rules of $\Pi$
    is called the ground instantiation of $\Pi$.
\end{definition}

\begin{example}\label{exp:grounding}
    The ground instantiation of the logic program $(\Sigma_1, \Pi_1)$ defined in the
    Example~\ref{exp:logic_program1} can be found as follows:

    The fact $\aspfact{p(True)}$ does not contain any variable nor function symbol.
    For this, it is ground. The same holds for the fact $\aspfact{q(5, True)}$
    and the rule $\asprule{p(False)}{\aspnot p(True)}$

    To ground the rule $\asprule{q(Y, True)}{q(X, (Y < X))}$,
    this rule should be first substitued for each rule with pair $(X, Y)$ substitued
    for every element from $\sigma_\NN \times \sigma_\NN$, that is every element
    from $\NN_0\times \NN_0$. Then in the second parameter of atom in the body,
    the expression substitued for $(Y < X)$ would be replaced with $True$ or $False$,
    depending on the substitution for $X$ and $Y$. This would lead to the ground
    instantiation $gr(\Sigma_1, \Pi_1)$ being infinite.

    To fight this, we can allow only the instantiation of rule $r$ in which
    only the substitutions of variables in the $\aspbody^+(r)$
    leading to the already existing atoms are allowed.
    After the stable model will be defined,
    it will be easy to see that this constraint does not remove any
    stable models of program, as every atom in it has to also be
    in the head of some rule.

    In this example, we already know that the atom $p(5, True)$ does exist.
    This means we can substitute $X\equiv 5$ and $(Y<X)\equiv True$.
    Also as $(Y<X)\equiv (Y < 5) \equiv True$, to successfully instantiate
    the rule, it has to hold that $Y < 5$. There are no other constraints
    on the rule, thus the rule instantiates into rules:

    \begin{equation*}
        \begin{matrix}
            Y=0 & : & \asprule{q(0, True)}{q(5, True)}\\
            Y=1 & : & \asprule{q(1, True)}{q(5, True)}\\
            Y=2 & : & \asprule{q(2, True)}{q(5, True)}\\
            Y=3 & : & \asprule{q(3, True)}{q(5, True)}\\
            Y=4 & : & \asprule{q(4, True)}{q(5, True)}\\
        \end{matrix}
    \end{equation*}

    But now we have introduced new atoms on predicate $q$. We have to also add
    all instances of the rule, in which these atoms are allowed.
    As the individual atoms only differ in the first term, the instances are
    seen easily. The instances added by the new rules also do not introduce
    new atoms on predicate $q$, the ground instance is thus in this case finite.

    \begin{equation*}
        \begin{matrix}
            X=4, Y=0 & : & \asprule{q(0, True)}{q(4, True)}\\
            X=4, Y=1 & : & \asprule{q(1, True)}{q(4, True)}\\
            X=4, Y=2 & : & \asprule{q(2, True)}{q(4, True)}\\
            X=4, Y=3 & : & \asprule{q(3, True)}{q(4, True)}\\
            X=3, Y=0 & : & \asprule{q(0, True)}{q(3, True)}\\
            X=3, Y=1 & : & \asprule{q(1, True)}{q(3, True)}\\
            X=3, Y=2 & : & \asprule{q(2, True)}{q(3, True)}\\
            X=2, Y=0 & : & \asprule{q(0, True)}{q(2, True)}\\
            X=2, Y=1 & : & \asprule{q(1, True)}{q(2, True)}\\
            X=1, Y=0 & : & \asprule{q(0, True)}{q(1, True)}\\
        \end{matrix}
    \end{equation*}

    The full ground instantiation $gr((\Sigma_1, \Pi_1))$ is thus the program:

    \begin{equation*}
        gr((\Sigma_1, \Pi_1)) =
            \begin{cases}
                \aspfact{p(True)}\\
                \aspfact{q(5, True)}\\
                \asprule{p(False)}{\aspnot p(True)}\\
                \asprule{q(0, True)}{q(5, True)}\\
                \asprule{q(1, True)}{q(5, True)}\\
                \asprule{q(2, True)}{q(5, True)}\\
                \asprule{q(3, True)}{q(5, True)}\\
                \asprule{q(4, True)}{q(5, True)}\\
                \asprule{q(0, True)}{q(4, True)}\\
                \asprule{q(1, True)}{q(4, True)}\\
                \asprule{q(2, True)}{q(4, True)}\\
                \asprule{q(3, True)}{q(4, True)}\\
                \asprule{q(0, True)}{q(3, True)}\\
                \asprule{q(1, True)}{q(3, True)}\\
                \asprule{q(2, True)}{q(3, True)}\\
                \asprule{q(0, True)}{q(2, True)}\\
                \asprule{q(1, True)}{q(2, True)}\\
                \asprule{q(0, True)}{q(1, True)}\\
            \end{cases}
    \end{equation*}
\end{example}

The Example~\ref{exp:grounding} also illustrates the need for the logic programs
to be well written. As was shown in this example, due to a single flawed rule,
the size of the ground instantiation of the logic program is quadratic
given the highest number in the first parameter of atoms of the predicate $q$.

Grounders differ in the exact algorithm of the grounding as well as in the
allowed signature of the logic program.
The exact full syntax and semantics
of the ASP grounder GRINGO can be found in~\cite{GEBSER_2015}.

% Semantics of a rule

\begin{definition}{Partial interpretation of a signature.}
    A partial interpretation $S$ of a signature $\Sigma$ is a set
    of $\Sigma$-literals in which for each $\Sigma$-atom $\varphi$
    there is either $\varphi$, $\neg\varphi$ or neither of them.
\end{definition}

% TODO: Make this better

A partial interpretation can be used to denote the solution of a logic program.
It allows for 3-valued logic --- an atom either is true, is false or is unknown.
The atom being unknown can specify an atom, that was not derived,
while the atom being false strictly says it cannot be true.

For a partial interpretation to be a solution (answer set) of the program,
it further needs to be consistent with the logic program
and all its literals have to be founded in the program (the set has to be minimal).

Each rule $r$ is seen as a statement $\asphead(r)$ holds
if all the literals from the $\aspbody^+(r)$ hold
and at the same time no literal from the $\aspbody^-(r)$ hold.
The facts have empty body, thus their head has to hold in every solution.
On the other hand, the body of constraints can never be consistent with
the solution as its head is empty thus can not be consistent with solution.
Further follows a more formal definition.

\begin{definition}{Consistent partial interpretation.}\label{consistent_interpretation}
    Partial interpretation $S$ is said to be consistent with the body of rule $r$
    if and only if $\aspbody^+(r) \subseteq S$ and
    $\aspbody^-(r) \cap S = \emptyset$.

    Partial interpretation $S$ is said to be consistent with the head of rule $r$
    if and only if $\asphead(r) \cap S \neq \emptyset$

    Partial interpretation $S$ is said to be consistent with a rule $r$
    if and only if $S$ is not consistent with $\aspbody(r)$
    or $S$ is consistent with $\asphead(r)$.

    Partial interpretation $S$ is said to be consistent with a logic program
    $(\Sigma, \Pi)$ if and only if it is consistent with all of its rules.
\end{definition}

\begin{definition}{Basic logic program.}%
    \label{def:basic_logic_program}
    A logic program $(\Sigma, \Pi)$ is called basic if there is no negative atom
    in body of any of its rules nor constraints, that is
    \begin{equation*}
        \forall r\in \Pi.\ \aspbody^{-}(r) = \emptyset
    \end{equation*}
\end{definition}

Now for the answer sets (partial interpretations that are solutions)
of a logic program a reasonable constraints
on them are to satisfy (be consistent with) the logic program and to only contain
information that needs to be true (each answer set has to be minimal).

% WTF? This does not work for the general logic programs?
%
% a; b; c.
% :- not a. :- not b. :- not c.
%
% Minimal consistent interpretation is {a, b, c}
% but by the definition 3.1.11 it is not an answer set.
\begin{definition}{Answer set of a basic logic program~\cite{KRHandbook}.}\label{def:as_basic}
    Let $(\Sigma, \Pi)$ be a basic logic program and $S$ be a partial interpretation
    of $\Sigma$. $S$ is an answer set (solution) for $\Pi$ if $S$ is minimal
    (in the sense of set-theoretic inclusion) among the partial interpretations
    consistent with $\Pi$.
\end{definition}

For a grounded basic logic program, finding its answer set (solution)
is easy. It can be built incrementally.
Heads of facts have to be included in the partial interpretation for it to be
an answer set. Then in each step all heads of rules, that have its
bodies consistent with the previous set, are added into the set.
This is done as long as any literal is added into the partial interpretation.

\begin{lemma}{Incremental building of the answer set of the basic logic program.}%
    \label{lemma:incremental}
    Let $(\Sigma, \Pi)$ be a basic logic program with at most one literal
    in the head of every rule.
    Let
    \begin{equation*}
        S_0 = \emptyset,
    \end{equation*}
    \begin{equation*}
        S_{i+1} = S_{i}\cup \{\varphi \mid \{\varphi\} = \asphead(r), \aspbody^+(r) \subseteq S_i\} \text{ for each } i \geq 0.
    \end{equation*}
    Denote by $S_\infty$ the union of all $S_i$ for $i\geq 0$.
    If it holds that:
    \begin{itemize}
        \item for every $\Sigma$-atom $\varphi$, $S_\infty$ contains at most one
            of literals $\varphi, \neg\varphi$, and
        \item $S_\infty$ is consistent with every constraint of $\Pi$,
    \end{itemize}
    then $S_\infty$ is
    the only answer set of program $(\Sigma, \Pi)$. Else $(\Sigma, \Pi)$ has no answer set.

    \begin{proof}
        $S_\infty$ does contain only (some of) heads of rules from $\Pi$,
        each of these being $\Sigma$-literals. If $S_\infty$ does also not contain
        both $\varphi$ and $\neg\varphi$ for any $\Sigma$-atom $\varphi$,
        $S_\infty$ is a partial interpretation of $\Sigma$.

        By definition $S_\infty$ contains the literal from the head of each rule,
        that has its body consistent with the $S_\infty$. For this the $S_\infty$
        is consistent with every non-constraint rule of the logic program.

        Let $rank: S_\infty \to \NN_0$ be a function that assigns to every literal
        $\varphi \in S_\infty$ the lowest value $n$ such that $\varphi \in S_n$.
        Let there be some answer set $A$ of $(\Sigma, \Pi)$ such that
        $S_\infty\setminus A = \Delta \neq \emptyset$.
        Let $\alpha$ be a literal from $\Delta$ with the lowest $rank$.
        By definition, $rank(\alpha) \geq 1$ as $S_0$ is empty.
        As the literal $\alpha\in S_{rank(\alpha)}$ and $\alpha \not\in S_{rank(\alpha)-1}$,
        either $\alpha$ is in a head of some fact (thus $rank(\alpha)=1$ and $\alpha$
        needs to be in $A$ for it to be consistent),
        or there must be some rule $r\in\Pi$ such that $\aspbody^+(r)$ is not consistent
        with $S_{rank(\alpha)-2}$ and is consistent with $S_{rank(\alpha)-1}$.
        (The rule $r$ is what made the literal $\alpha$ in $S_{rank(\alpha)}$.)
        But $A$ does contain all the literals from $S_{rank(\alpha)-1}$. (Literal $\alpha$
        has the lowest rank of all literals from $S_\infty$ not included in $A$.)
        Thus $A$ is not consistent with $r$ and is not an answer set of $(\Sigma, \Pi)$.
        By contradiction, for every answer set $B$ of $(\Sigma, \Pi)$,
        $S_\infty \subseteq B$. As the answer set is by definition minimal,
        the $S_\infty$ is the only possible answer set of $(\Sigma, \Pi)$.
    \end{proof}
\end{lemma}

\begin{example}
    Let $(\Sigma, \Pi)$ be a logic program,
    \begin{equation*}
        \Pi =
            \begin{cases}
                \aspfact{p(a)}\\
                \asprule{p(b)}{p(a)}\\
                \asprule{p(c)}{p(b)}\\
                \aspfact{p(x)}\\
                \asprule{p(y)}{p(x), p(a)}\\
                \asprule{p(z)}{p(x), p(c)}\\
                \asprule{p(n)}{p(a), p(k)}\\
            \end{cases}
    \end{equation*}

    When building the solution, we first start with an empty set.
    \begin{equation*}
        S_0 = \{\}
    \end{equation*}
    In the first step only the facts have bodies consistent with
    the partial solution $S_0$. We can add literals in their heads into
    the partial solution.
    \begin{equation*}
        S_1 = \{p(a), p(x)\}
    \end{equation*}
    With these two literals in the partial solution, rules $\asprule{p(b)}{p(a)}$
    and $\asprule{p(y)}{p(x), p(a)}$ have their bodies consistent
    with the partial solution $S_1$. For this we add literals in heads of these
    rules into the partial solution.
    \begin{equation*}
        S_2 = \{p(a), p(x), p(b), p(y)\}
    \end{equation*}
    In the next step, another rule $\asprule{p(c)}{p(b)}$ has its body consistent
    with the partial solution $S_2$. We add the literal from its head into
    the partial solution.
    \begin{equation*}
        S_3 = \{p(a), p(x), p(b), p(y), p(c)\}
    \end{equation*}
    This addition into the partial solution makes the body of yet another
    rule $\asprule{p(z)}{p(x), p(c)}$ to be consistent with the partial solution.
    \begin{equation*}
        S_4 = \{p(a), p(x), p(b), p(y), p(c), p(z)\}
    \end{equation*}
    No other literals have to be added into the partial solution.
    $S_4$ is the only solution of logic program $(\Sigma, \Pi)$.
\end{example}

The only literal from the heads of rules that have not been included
into the partial solution is $p(n)$. This literal relies
on some unknown literal $p(k)$ that is not in any head of rules of $\Pi$.
Another case of literal that is not in a solution of logic program
is shown in the following example.

\begin{example}\label{example:cycle}
    Let $(\Sigma, \Pi)$ be a logic program,
    \begin{equation*}
        \Pi =
            \begin{cases}
                \asprule{p(a)}{p(b)}\\
                \asprule{p(b)}{p(a)}\\
            \end{cases}
    \end{equation*}
    This logic program a single solution, that is an empty set.
    While the partial interpretation $\{p(a), p(b)\}$ is consistent
    with the logic program, it is not the minimal set.
\end{example}

% TODO: The negative literals here are not a good way to say it
The simple way of building answer sets of a basic programs,
can be extended to all logic programs. In basic programs we have assumed
that no rule with negative literals does exist in the logic program.
To fight the negative literals in a general extended logic program,
we can first define a partial interpretation of $\Sigma$ that constitutes
the possible answer set, then take all rules $r\in\Pi$ that have no intersection
of their $\aspbody^-(r)$ with our partial interpretation and into the reduced logic program
include only the nonnegative $\aspbody^+(r)$ from them. If this reduced program
yields the same answer set as the partial interpretation we have defined,
it is an answer set.

\begin{definition}{Reduct of a logic program~\cite{KRHandbook}.}
    Let $(\Sigma, \Pi)$ be a logic program, $S$ be a partial interpretation of $\Sigma$.
    Let $\Pi^S$ be a set of rules such that
    \begin{equation*}
        \Pi^S = \{\asprule{\asphead(r_i)}{\aspbody^+(r_i)}
                  \mid r_i\in \Pi,\,\aspbody^-(r_i)\cap S = \emptyset\}
    \end{equation*}
    Then $\Pi^S$ is called a reduct of $\Pi$ relative to the partial interpretation $S$.
\end{definition}

\begin{example}
    In the reduct can be encountered a new type of rule that was not yet discussed here.
    Let $\Pi = \{\asprule{}{\aspnot \varphi}\}$, $S = \{\}$ for some atom $\varphi$.
    The reduct $\Pi^S$ contains
    a single rule with both head and body empty.
    Based on the Definition~\ref{consistent_interpretation},
    no partial interpretation can be consistent
    with such rule as no partial interpretation can be consistent with an empty head
    and every partial interpretation is consistent with an empty body.
    This is however a desired behaviour, the interpretation $S$ was defined
    in a way it was not consistent with the rule so it is only right
    the reduct can not have any answer set.
\end{example}

\begin{definition}{Answer set of general grounded logic program~\cite{KRHandbook}.}%
    \label{answer_set_general}
    A partial interpretation $S$ of $\Sigma$ is an answer set for $(\Sigma, \Pi)$
    if $S$ is an answer set for $\Pi^S$.
\end{definition}

Definition~\ref{answer_set_general} gives a direct way of computing answer sets
of any logic programs. For every possible subset $S$ of grounded atoms in program $\Pi$,
the reduct $\Pi^S$ can be constructed and evaluated for its answer set.
If the found answer set is equal to the subset of grounded atoms $S$, it is said
to be also an answer set of the program $\Pi$.

Let's illustrate the computation of an answer set of general logic program on two examples.

\begin{example}
    Let $(\Sigma, \Pi)$, where
    \[\Pi = \{\asprule{\varphi}{\varphi}, \asprule{\psi}{\aspnot \varphi}\},\]
    be a logic program with two atoms $\varphi, \psi$.
    There are 4 subsets of set of all atoms. First the reduct relative to
    the subset is made, on it the calculation of its answer set (AS) is made.
    If the answer set of reduct is equal to the subset, it is an answer set of the
    whole logic program.
    \begin{center}
        \begin{tabular}{L L L}\toprule{}%
            S        & \Pi^S          & \text{AS of }\Pi^S \\\midrule{}%
            \{\}     & \asprule{\varphi}{\varphi} & \{\psi\} \\
                     & \asprule{\psi}{}  &       \\\addlinespace[0.5em]
            \{\varphi\}    & \asprule{\varphi}{\varphi} & \{\}  \\\addlinespace[0.5em]
            \{\psi\}    & \asprule{\varphi}{\varphi} & \{\psi\} \\
                     & \asprule{\psi}{}  &       \\\addlinespace[0.5em]
            \{\varphi, \psi\} & \asprule{\varphi}{\varphi} & \{\}  \\
            \bottomrule{}
        \end{tabular}
    \end{center}
    There is only a single answer set of $(\Sigma, \Pi)$, that is $\{\psi\}$.
    Similiar to Example~\ref{example:cycle}, in the reduct $\Pi^{\{\varphi\}}$
    partial interpretation $\{\varphi\}$ is not an answer set as it is
    not minimal.
\end{example}

\begin{example}\label{example:xor}
    Let $(\Sigma, \Pi)$, where
    \[\Pi = \{\asprule{\varphi}{\aspnot \psi}, \asprule{\psi}{\aspnot \varphi}\},\]
    be a logic program with two atoms $\varphi, \psi$.
    Again, there are 4 subsets of set of all atoms.
    \begin{center}
        \begin{tabular}{L L L}\toprule{}%
            S        & \Pi^S         & \text{AS of }\Pi^S \\\midrule{}%
            \{\}     & \asprule{\varphi}{} & \{\varphi, \psi\} \\
                     & \asprule{\psi}{} &          \\\addlinespace[0.5em]
            \{\varphi\}    & \asprule{\varphi}{} & \{\varphi\}    \\\addlinespace[0.5em]
            \{\psi\}    & \asprule{\psi}{} & \{\psi\}    \\\addlinespace[0.5em]
            \{\varphi, \psi\} &               & \{\}     \\
            \bottomrule{}
        \end{tabular}
    \end{center}
    This time there are two answer sets of $(\Sigma, \Pi)$, $\{\varphi\}$ and $\{\psi\}$.
\end{example}

% Well, this does not work
% \subsection{Disjunctive heads of rules}
% 
% Until now only rules with single or none literal in the head were discussed.
% In the Definition~\ref{def:rule}, the rule has been defined with
% possibly multiple literals in the head. However when we were computing
% the answer sets of logic programs, we assumed the heads to be limited.
% While the Definition~\ref{def:as_basic} does remain unchanged,
% computing 
% 
% The logic program of the Example~\ref{example:xor} introduced a useful property.
% Its two rules assert that at least a single of $\varphi, \psi$
% is always included into the answer set of this logic program.
% This property can be used to rewrite each rule with multiple literals in head
% into multiple rules, each with a single literal in its head.
% 
% \begin{lemma}{Rule with disjunctive head.}
%     Let $(\Sigma, \Pi)$ be a logic program, containing rule $r$ with head
%     \[\asphead(r) = \{l_1, l_2, l_3, \ldots, l_k\}\]
%     and body
%     \[\aspbody(r) = \{l_{k+1}, \ldots, l_m, \aspnot l_{m+1}, \ldots, \aspnot l_{n}\}\]
%     Then it is equivalent to program $(\Sigma, \Pi')$ where $\Pi'$ is
%     modified set of rules:
%     \begin{equation*}
%         \begin{matrix}
%             {r'}_1 = (\asprule{l_1}{\aspnot l_2, \aspnot l_3, \ldots, \aspnot l_k, l_{k+1}, \ldots, l_m, \aspnot l_{m+1}, \ldots, \aspnot l_{n}})\\
%             {r'}_2 = (\asprule{l_2}{\aspnot l_1, \aspnot l_3, \ldots, \aspnot l_k, l_{k+1}, \ldots, l_m, \aspnot l_{m+1}, \ldots, \aspnot l_{n}})\\
%             \vdots\\
%             {r'}_k = (\asprule{l_k}{\aspnot l_1, \aspnot l_2, \ldots, \aspnot l_{k-1}, l_{k+1}, \ldots, l_m, \aspnot l_{m+1}, \ldots, \aspnot l_{n}})\\
%         \end{matrix}
%     \end{equation*}
%     \[R' = \{{r'}_1, {r'}_2, \ldots, {r'}_k\}\]
%     \[\Pi' = (\Pi \setminus\{r\})\cup R'\]
% \end{lemma}
