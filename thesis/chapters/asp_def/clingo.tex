\section{Clingo}

Clingo is an integrated ASP system, consisting of a grounder Gringo and solver
Clasp~\cite{aspEasy2016}. In the following section I will show basics of the
Clingo language, Gringo as translation from Clingo to Aspif language and
solving with Clasp.

\subsection{Clingo language}
Clingo language~\cite{gebser2019potassco} is used for transcribing ASP programs
in the Clingo system.
The Clingo language does allow for facts, rules and constraints just like
the logic programs defined in this thesis.

\begin{example}
    Transcription of a logic program from the Example~\ref{exp:long_program} to Clingo language.
    On the left side is a set of rules from this example,
    on the right is equivalent program in the Clingo language.

    \begin{minipage}{.4\textwidth}
    \begin{equation*}
        \Pi =
            \begin{cases}
                \aspfact{p(a)}\\
                \asprule{p(b)}{p(a)}\\
                \asprule{p(c)}{p(b)}\\
                \aspfact{p(x)}\\
                \asprule{p(y)}{p(x), p(a)}\\
                \asprule{p(z)}{p(x), p(c)}\\
                \asprule{p(n)}{p(a), p(k)}\\
            \end{cases}
    \end{equation*}
    \end{minipage}
    \hfill
    \begin{minipage}{.4\textwidth}
        \vspace{1em}
        \begin{lstlisting}[language=Prolog, numbers=none, basicstyle=\linespread{1.25}\normalsize]
p(a).
p(b) :- p(a).
p(c) :- p(b).
p(x).
p(y) :- p(x), p(a).
p(z) :- p(x), p(c).
p(n) :- p(a), p(k).
\end{lstlisting}
    \end{minipage}
\end{example}

\subsection{Basic syntax of Clingo language}

\vspace{-0.5em}
\textit{%
    Unless specified otherwise,
    source of information in this section is~\cite{gebser2019potassco}.}
    \vspace{1em}

Terms in the Clingo language are of multiple sorts.
The simplest sort of terms are constants and strings.
Constants match the regular expression \texttt{\_*[a-z][A-Za-z0-9\_']*}.  % chktex 8
They start with underscores followed by lowercase letter and a sequence of characters.
Similiarly, strings match the regular expression
\texttt{"([\^{}\textbackslash"↵]|\textbackslash[\textbackslash"n])*"}.  % chktex 18
They allow for any reasonable text.

Another sort of Clingo language signature are integers.
Constant integers can be captured by the regular expression
\texttt{-?(0|[1-9][0-9]*)}.  % chktex 8
As can be seen, zero character can only be at the start of zero number.
Number zero (\texttt{0}) and negative zero (\texttt{-0}) are the same constants.
The range of integers is platform dependent, they are either 32-bit or 64-bit.

Grounder Gringo also built-in support for arithmetic functions over integers.
Following symbols are used: \texttt{+} (addition), \texttt{-} (substraction), \texttt{*}
(multiplication), \texttt{/} (integer division, rounding down), \texttt{\textbackslash}
(modulo), \texttt{**} (exponentiation), \texttt{|\ldots|} (absolute value),
\texttt{\&} (bitwise AND), \texttt{?} (bitwise OR), \texttt{\^{}} (bitwise XOR),
and \texttt{\~} (bitwise complement).  % chktex 14
Arithmetic functions use infix notation.

Gringo also allows for comparison predicates over all terms.
These are predicates that are evaluated during grounding.
The comparison predicates are \texttt{=} (equal), \texttt{!=} (not equal),
\texttt{<} (less than), \texttt{<=} (less than or equal),
\texttt{>} (greater than) and \texttt{>=} (greater than or equal).
Integers are comared in the usual way, constants and strings are ordered lexicographically.
All integers are smaller than constants which are smaller than strings.
Additionaly two special constants \texttt{\#sup} and \texttt{\#inf} do exist,
constituting to the greatest and lowest term.

Another sorts of terms are functions and tuples.
Such terms take form of \texttt{$f$($t_1$, \ldots, $t_n$)} for the functions
and \texttt{($t_1$, \ldots, $t_n$)} for the tuples where $f$ is a function symbol
and $t_1$, \ldots, $t_n$ are terms. The tuple has to have at least a single term long,
in that case it is written as \texttt{($t_1$,)} to differentiate it from brackets.
Function symbols are constrained in the same way as names of constants.
Function with no terms as arguments is by the grounder seen as a constant with the function
symbol as a name.
Functions and tuples as terms do not have any interpretation.
There is no mapping from function terms to some other terms as the name might suggest.

Finally, the last type of terms are variables.
Variables can be matched with a regular expression \texttt{\_*[A-Z][A-Za-z0-9\_']*}.  % chktex 8
Variables are used to generalize rules of the logic program,
a ground (variable-free) program is constructed during the process of grounding.
A special variable \texttt{\_} is called an annonymous variable or wildcard.
At the start of the grounding process, a fresh (not used elsewhere) variable term
is substitued for each annonymous variable.
It stands for a placeholder, meaning any term can take its place.

An atom is syntatically consistent with a constant or a function.
The predicate symbol starts with any number of underscores \texttt{\_}
followed by a lowercase letter and any number of alphanumeric characters,
underscores and \texttt{'}.
Atom using predicate symbol with arity of lenght zero can be written without the parentheses.
One predicate symbol can be used with different arities.

A literal is an atom or its classical negation.
In the Clingo language, classical negation is indicated by minus symbol \texttt{-}  % chktex 8
before the literal.

Finally, each ASP rule $r$ of form
\[\asprule{l_0\, or\, \ldots\, or\, l_k}{l_{k+1}, \ldots, l_{m}, \aspnot l_{m+1}, \ldots, \aspnot l_{n}}\]
is transcribed into following expression of Clingo language:
\begin{center}
    \texttt{$l_0$; \ldots; $l_k$ :- $l_{k+1}$, \ldots, $l_{m}$, not $l_{m+1}$, \ldots, not $l_{n}$.}  % chktex 26
\end{center}
Specially, if $\asphead(r)$ is nonempty and $\aspbody(r)$ is empty, a shorthand transcription
can be used:
\begin{center}
    \texttt{$l_0$; \ldots; $l_k$.}  % chktex 26
\end{center}

In the basic notation of rules, semicolons \texttt{;} and commas \texttt{,}
can be freely interchanged. While this is true, I will try to stick to semicolons in heads
and commas in bodies of rules for better readability.
Where this is not possible (see Conditional literals), I will always point it out.

\begin{example}\label{exp:transcription_clingo}%
    Transcription of terms and atoms of Example~\ref{Sigma1}.

    In the example, signature $\Sigma_1$ had sorts
    \begin{equation*}
        \sigma_\BB = \{True, False\}, \sigma_\NN = \NN_0.
    \end{equation*}
    On these sorts, function symbols $\vee, \wedge$ of arity
    $\sigma_\BB\times \sigma_\BB\to \sigma_\BB$,
    and function symbols $=, <$ of arity $\sigma_\NN\times\sigma_\NN\to \sigma_\BB$
    were defined.
    Also, there were two predicate symbols, $p$ of arity $\sigma_\BB$,
    and $q$ of arity $\sigma_\NN\times\sigma_\BB$.
    Finally, there were 3 variables $X, Y, Z$ of sort $\sigma_\NN$.

    Let the interpretation of these function symbols be the usual one,
    that is $\vee, \wedge$ are logical \textit{OR} and \textit{AND}
    and $=, <$ are \textit{equal} and \textit{less than}.
    In Clingo language, there is no such thing as boolean values. However, we can still use
    integers $0$ and $1$ in their place. Function symbols $\vee$ and $\wedge$
    can then be modelled by using bit and and bit or arithmetic functions of the Clingo language.

    Function symbols $=$ and $<$ are a bit more problematic.
    In the Clingo language as defined so far there are no user defined function\footnote{
        In fact there are External functions which use Python or Lua API
        to compute values on grounding.
        You can read about them in~\cite{gebser2019potassco}.
        I will not be showing them here as I do not use them in my implementation.
    }.
	% TODO: Rewrite this
    When working with rules with these function symbols, we can however use
    comparison predicates to build multiple rules, each with its own assignment
    to the term value.

    In the following, on the left side are $\Sigma$-terms of sort $\sigma_\BB$,
    on the right side are their transcriptions into the Clingo language according to
    the above text. % TODO: additional predicates
	When the expression would need to be rewritten into multiple rules,
    different cases are inside curly brackets split by comma, for each case the substituted term
    is before the square brackets and added comparison predicates are inside the square brackets.

    \begin{minipage}{.45\textwidth}
        $True,\ False$,\\
        $True \vee False$, $False \wedge True$,\\
        $(42 = 42)$,\\
        \\
		$(X = 17)$,\\
        \\
        $(3 < Y) \vee (6 = 6)$,\\
        \\
        \\
        \\
        $[(5 = 1)\wedge True] \vee False$\\
    \end{minipage}
    \hfill
    \begin{minipage}{.5\textwidth}
        \begin{lstlisting}[language=Prolog, numbers=none, basicstyle=\normalsize]
1, 0,
1 ? 0, 0 & 1,
{1 [42 = 42],
 0 [42 != 42]},
{1 [X = 17],
 0 [X != 17]},
{1 ? 1 [3 < Y, 6 = 6],
 1 ? 0 [3 < Y, 6 != 6],
 0 ? 1 [3 >= Y, 6 = 6],
 0 ? 0 [3 >= Y, 6 != 6]},
{(1 & 1) ? 0 [5 = 1],
 (0 & 1) ? 0 [5 != 1]}
\end{lstlisting}
    \end{minipage}

    Usage of these and other similiar $\Sigma$-terms in atoms is described
	in the following transcription.
    On the left side are some $\Sigma$-facts, on the right side is their transcription into
    the Clingo language.

    \begin{minipage}{.45\textwidth}
        $\aspfact{p(True)}$\\
        $\aspfact{p(False)}$\\
        $\aspfact{p([(5 = 1) \wedge True] \vee False)}$\\
		\\
        $\aspfact{q(0, True)}$\\
        $\aspfact{q(123, (42 = 42))}$\\
    \end{minipage}
    \hfill
    \begin{minipage}{.5\textwidth}
        \begin{lstlisting}[language=Prolog, numbers=none, basicstyle=\normalsize]
p(1).
p(0).
p((1 & 1) ? 0) :- 5 = 1.
p((0 & 1) ? 0) :- 5 != 1.
q(0, 1).
q(123, 1) :- 42 = 42.
q(123, 0) :- 42 != 42.
\end{lstlisting}
    \end{minipage}
\end{example}

When transcripting rules of Example~\ref{exp:transcription_clingo},
we have made some facts into rules by introducing new comparison predicates into the rule.
In the section about grounding, we will however see, that these predicates will be
removed during the grounding process.

\subsection{Extensions of ASP in Clingo language}

\textit{%
    Unless specified otherwise,
    source of information in this section is~\cite{gebser2019potassco}.}
    \vspace{1em}

Clingo language allows for many extensions of the Answer set programming.
In this section, I will describe extensions, which will be used in the implemention part
of this thesis.

% TODO: Add this into the grounding section

\subsubsection{Intervals and Pooling}

When defining rules, it is often desired to define rules on a range of values.
For this type of rule, Clingo language allows for the use of intervals and pools.
Intervals and pools can be used in place of terms in both heads and bodies of rules.

For two whole numbers $i, j$, an interval \texttt{$i$..$j$} means,
that during grounding a rule containing this interval expands to multiple rules,
one for each $k$ s.t.\ $i \leq k\leq j$ in its place.
If there are multiple such intervals, they expand one by one,
latter ones expanding in all rules already partially expanded by previous intervals.

\begin{example}
    Consider the following program in Clingo language~\cite{gebser2019potassco}:
    \begin{lstlisting}[language=prolog, numbers=none]
grid(1..3, 1..3).
\end{lstlisting}
    At the start of the grounding process, this program expands into the program:
    \begin{lstlisting}[language=prolog, numbers=none]
grid(1, 1). grid(1, 2). grid(1, 3).
grid(2, 1). grid(2, 2). grid(2, 3).
grid(3, 1). grid(3, 2). grid(3, 3).
\end{lstlisting}
\end{example}

A pool is an expression in the form \texttt{($t_1$; $t_2$; \ldots; $t_n$)}.
Similiarly to the intervals during the grounding of a rule containing this pool,
the rule expands to all rules containing $t_1$, $t_2$, \ldots, $t_n$ in place
of this pool. If there are multiple pools they expand consecutively.

\begin{example}
    Consider the following program in Clingo language:
    \begin{lstlisting}[language=prolog, numbers=none]
grid((a; 3; #inf), (1; 2)).
\end{lstlisting}
    At the start of the grounding process, this program expands into the program:
    \begin{lstlisting}[language=prolog, numbers=none]
grid(a, 1).    grid(a, 2).
grid(3, 1).    grid(3, 2).
grid(#inf, 1). grid(#inf, 2).
\end{lstlisting}
\end{example}

\begin{example}
    Consider the following program in Clingo language:
    \begin{lstlisting}[language=prolog, numbers=none]
stairs(A, 1..A) :- A = 1..3.
\end{lstlisting}
	When grounding this program, the grounder has to first expand the interval in the body.
    The head interval contains a variable \texttt{A} with value not yet known.
    After the first expansion, the following program emerges.
    \begin{lstlisting}[language=prolog, numbers=none]
stairs(A, 1..A) :- A = 1.
stairs(A, 1..A) :- A = 2.
stairs(A, 1..A) :- A = 3.
\end{lstlisting}
    Now there is only a single possible value of \texttt{A} in each expanded rule.
    \texttt{A} can be unified with this value and the head interval can be expanded.
    For better readability, I have removed comparison predicates from the body.
    \begin{lstlisting}[language=prolog, numbers=none]
stairs(1, 1).
stairs(2, 1). stairs(2, 2).
stairs(3, 1). stairs(3, 2). stairs(3, 3).
\end{lstlisting}
\end{example}

\subsubsection{Conditional literals}

A conditional literal is an expression of the form:
\begin{center}
    \texttt{$l_0$ : $l_1$, \ldots, $l_n$}  % chktex 26
\end{center}
where each of $l_0, l_1, \ldots, l_n$ are literals. This expression has a meaning
of an inner implication. The conditional literal is consistent with a signature
if and only if its head $l_0$ is consistent with this signature
or its body $l_1, \ldots, l_n$ is not consistent with this signature.
It can be seen as an expression
\textit{put $l_0$ in its place if all literals $l_1, \ldots, l_n$
are consistent with the partial solution}.

In the notation of conditional literals, use of commas \texttt{,} is compulsory.
This is different from the rules, where commas and semicolons can be interchanged freely.
When the conditional literal is not the last literal of a rule body,
a semicolon must be used as an end of this conditional literal.
Conditional literals can be quite nontrivial, especially when used in head of a rule.

\begin{example}\label{exp:conditional_literal}
    Consider following rules in Clingo language:
    \begin{lstlisting}[language=prolog, numbers=none]
a :- b : c; d.
a :- b : c, d.
\end{lstlisting}

    The first rule has a body composed of a conditional literal
    \texttt{b : c} and a literal \texttt{d}.  % chktex 26
    For this rule an equivalent program without conditional literals would be following:
    \begin{lstlisting}[language=prolog, numbers=none]
x :- b.
x :- not c.
a :- x, d.
\end{lstlisting}
    In this logic program the conditional literal was replaced with a fresh literal \texttt{x}.
    This literal is true if either the head of conditional literal is consistent
    or the body is not consistent.

    The only difference between first and second rule is the comma or semicolon
    between literals \texttt{c} and \texttt{d}.
    Second rule has a body consisting only of conditional literal \texttt{b : c, d}.  % chktex 26
    An equivalent program to this rule would be following:
    \begin{lstlisting}[language=prolog, numbers=none]
x :- b.
x :- not c.
x :- not d.
a :- x.
\end{lstlisting}
\end{example}

\begin{example}
    Consider following rule in Clingo language:
    \begin{lstlisting}[language=prolog, numbers=none]
a : b.
\end{lstlisting}
    The rule consists of a single conditional literal in the head.
    If the literal \texttt{b} is not consistent with the partial solution,
    this rule acts as a rule with both head and body empty, thus invalidating any such solution.
    Rather than simply rewriting as a rule with \texttt{a} and \texttt{b}
    in the head and body respectively, the rule is equivalent to following program:
    \begin{lstlisting}[language=prolog, numbers=none]
:- not b.
a :- b.
\end{lstlisting}
\end{example}

\subsubsection{Aggregates}

Aggregates are expressions of form
\begin{center}
    \texttt{$s_1$ $<_1$ $\alpha$ \{ $t_1$:$L_1$; \ldots; $t_n$:$L_n$ \} $<_2$ $s_2$},  % chktex 26
\end{center}
where $s_1, s_2$ are terms, $<_1, <_2$ are comparison predicate symols,
$\alpha$ is an aggregate symbol,
$t_1, \ldots, t_n$ are terms and $L_1, \ldots, L_n$ are sets of literals.

The full expression can be split into multiple parts. First, there is set of (conditional) terms%
\footnote{To distinguish between a comparison predicate symbol of Clingo language
and a equality in the normal means, in this subsection I will
use $\equiv$ rather than $=$ for the second.}
\begin{center}
    $T \equiv $ \texttt{\{ $t_1$:$L_1$; \ldots; $t_n$:$L_n$ \}}.  % chktex 26
\end{center}
This set can be computed as a union of subsets
\[T \equiv \{t_1:L_1\}\cup \ldots \cup \{t_n:L_n\}.\]
If any of these subsets contains variables, it can be grounded similiarly to rule.
Grounded expressions can then be evaluated into sets of terms.
The full expression is thus evaluated into set of terms containing all terms whose
sets of literals are consistent with the partial solution.

Set $T$ is evaluated in a set-like manner meaning if multiple its terms are the same,
only a single one of them is included in the final set.
When multiple occurences of the same term is needed, special form of terms can be used.
This form is tuple-like, only without outer brackets.

Should some term $t_i$ be included in every such set regardless of consistency of any atoms
(that is $L_i = \emptyset$),
both the colon and part with literals \texttt{:$L_i$} may be omitted.

\begin{example}\label{exp:aggregate_expressions}
    Let $S$ be a partial solution s.t.\ literals \texttt{p(1)}, \texttt{p(4)},
    \texttt{p(5)} and \texttt{p(7)} are consistent with $S$.

    The expression
    \begin{center}
        $T \equiv$ \texttt{\{ 1; 3; 4; a; x \}}  % chktex 26
    \end{center}
    does not contain any variables nor literals, thus it is evaluated onto itself.

    The expression
    \begin{center}
        $T \equiv$ \texttt{\{ 1:p(1); 3:p(3); 4:p(4), p(7) \}}  % chktex 26
    \end{center}
    does contain literals. When evaluated, it is first looked at the literals if they are
    consistent with the partial solution. For the first conditional term,
    \texttt{p(1)} is consistent with $S$, thus \texttt{1} is included in the set.
    The second conditional term depends on literal \texttt{p(3)} which is not consistent
    with $S$, \texttt{3} is not included in the set.
    Finally, the last conditional term depends on literals \texttt{p(4)} and \texttt{p(7)}.
    Both of these literals are consistent with $S$, \texttt{4} is included in the set.
    \begin{center}
        $T \hookrightarrow$ \texttt{\{ 1; 4 \}}  % chktex 26
    \end{center}

    The expression
    \begin{center}
        $T \equiv$ \texttt{\{ Y-X:p(X), p(Y), X <= Y \}}  % chktex 26
    \end{center}
    does contain both literals and variables. First we can start by grounding the expression.
    We are looking for pairs of literals \texttt{p(X), p(Y)} that comply to the comparison
    predicate \texttt{X <= Y}.
    \begin{center}
        $T\hookrightarrow$ \texttt{\{ 0:p(1), p(1); 3:p(1), p(4); 4:p(1), p(5); 6:p(1), p(7); 0:p(4), p(4); 1:p(4), p(5); 3:p(4), p(7); 0:p(5), p(5); 2:p(5), p(7); 0:p(7), p(7) \}}  % chktex 26
    \end{center}
    Now as we only took tuples of literals satisfying both the rules from the expression
    and being consistent with the partial solution, to evaluate $T$ we can directly
    take terms in the set.
    \begin{center}
        $T\hookrightarrow$ \texttt{\{ 0; 1; 2; 3; 4; 6 \}}  % chktex 26
    \end{center}

    In the evaluation of the last expression, we have lost the multiplicity of each term
    as it was evaluated as a set. To counteract that, we can add the lower parameter
    of each pair of literals to the terms.
    \begin{center}
        $T \equiv$ \texttt{\{ Y-X,X:p(X), p(Y), X <= Y \}}\\  % chktex 26
        $T\hookrightarrow$ \texttt{\{ 0,1; 3,1; 4,1; 6,1; 0,4; 1,4; 3,4; 0,5; 2,5; 0,7 \}}.  % chktex 26
    \end{center}
\end{example}

An aggregate acts as a function over set of conditional terms,
aggregate symbol $\alpha$ specifies the type of function.
Currently allowed aggregate symbols are \texttt{\#count},
\texttt{\#sum}, \texttt{\#sum+}, \texttt{\#min} and \texttt{\#max}.
These stand for count of different terms, sum over terms, sum over positive terms,
minimal term and maximal term respectively.
Max and min aggregates do follow the same ordering as comparison predicates.
If a term is composed of multiple parts, \texttt{\#sum} and \texttt{\#sum+}
do their summation over
the first part of term. If this part is not integer, the whole term is skipped.
Summation over tuples or functions is not implemented.
If the aggregate symbol $\alpha$ is omitted, it defaults to \texttt{\#count}.

Aggregate returns a value in the form of a term. This value can then be compared
with another term or assigned to variable. The same comparison predicate symbols
can be used in place of $<_1, <_2$ as in comparison predicates.
Both \texttt{$s_1$ $<_1$} and \texttt{$<_2$ $s_2$} can be omitted.
If only $<_1$ or $<_2$ is omitted, comparison predicates defaults to \texttt{<=}.

\begin{example}
    In this example we will work with expressions from Example~\ref{exp:aggregate_expressions}.
    Let $S$ be a partial solution s.t.\ literals \texttt{p(1)}, \texttt{p(4)},
    \texttt{p(5)} and \texttt{p(7)} are consistent with $S$.

    The rule
    \begin{center}
        \texttt{a(X) :- X = \#count\{ 1; 3; 4; a; x \}.}  % chktex 26
    \end{center}
    adds a single atom \texttt{q(5)} into the solution as all
    elements of the aggregate set are terms.
    Here only a single left-hand comparison predicate is used. The right-hand
    predicate is omitted.

    The rule
    \begin{center}
        \texttt{a :- 2 \#sum\{ 1:p(1); 3:p(3); 4:p(4), p(7) \} <= 7}  % chktex 26
    \end{center}
    adds a single atom \texttt{q} into the solution.
    As already shown, the inner expression evaluates into
    \begin{center}
        \texttt{\{ 1; 4 \}},  % chktex 26
    \end{center}
    sum of this set is equal to 5, which satisfies comparison \texttt{2 <= 5 <= 7}.
    The left-hand side comparison predicate symbol is omitted, thus by default is \texttt{<=}.

    Both rules
    \begin{center}
        \texttt{a(X) :- X = \#max\{ Y-X:p(X), p(Y), X <= Y \}.}  % chktex 26
    \end{center}
    and
    \begin{center}
        \texttt{a(X) :- X = \#max\{ Y-X,X:p(X), p(Y), X <= Y \}.}  % chktex 26
    \end{center}
    add a single atom \texttt{a(6)} into the solution. If tuple without brackets
    is used in the definition of term, the \texttt{\#max} aggregate does take only
    the first element into account.

    If both terms are needed, one can use the following rule:
    \begin{center}
        \texttt{a(X, Y) :- (X, Y) = \#max\{ (Y-X,X):p(X), p(Y), X <= Y \}.}  % chktex 26
    \end{center}
    This time each term of the aggregate set is a single tuple.
    As written in Basic syntax, terms are compared in lexicographic manner.
\end{example}

A special form of aggregates is also allowed in the head of a rule.
This type of aggregates, called head aggregates, are of form:
\begin{center}
    \texttt{$s_1$ $<_1$ $\alpha$ \{ $t_1$:$l_1$:$L_1$; \ldots; $t_n$:$l_n$:$L_n$ \} $<_2$ $s_2$},  % chktex 26
\end{center}
where $s_1, s_2$ are terms, $<_1, <_2$ are comparison predicate symols,
$\alpha$ is an aggregate symbol,
$t_1, \ldots, t_n$ are terms, $l_1, \ldots, l_n$ are literals
and $L_1, \ldots, L_n$ are sets of literals.

When evaluating the set, term $t_i$ is included in the set if both literal $l_i$
and all literals from $L_i$ are consistent with the solution.
Additionaly, if the aggregate forms a head of rule $r$ and body of $r$ is consistent
with the solution, the aggregate can introduce any number of literals $l_i$
for such all literals $L_i$ are consistent with the solution.
This is especially useful for generating an input space of logic program.

For all $s_1, s_2, <_1, <_2, \alpha, L_i$ the default values and rules for omitting
are still the same as when using the aggregates in the body.
Specially, the expression
\begin{center}
    \texttt{$s_1$ $<_1$ \{ $l_1$:$L_1$; \ldots; $l_n$:$L_n$ \} $<_2$ $s_2$},  % chktex 26
\end{center}
is equivalent to
\begin{center}
    \texttt{$s_1$ $<_1$ \#count\{ $t_1$:$l_1$:$L_1$; \ldots; $t_n$:$l_n$:$L_n$ \} $<_2$ $s_2$},  % chktex 26
\end{center}
where all $t_i$ are distinct.

\begin{example}\label{exp:input_space}
    Let the logic program consist of a single fact
    \begin{center}
        \texttt{\{a; b; c; d; e\}.}
    \end{center}
    This fact is a syntactic shortcut of fact\footnote{
        The following expression is not a valid rule in clingo language
        as it does not contain $\emptyset$. I choose to include it into the
        notation here to improve readability.
    }
    \begin{center}
        \texttt{\#count\{ 1:a:$\emptyset$; 2:b:$\emptyset$; 3:c:$\emptyset$; 4:d:$\emptyset$; 5:e:$\emptyset$ \}.}  % chktex 26
    \end{center}
    As an empty set of literals is consistent with any solution, any of atoms
    \texttt{a}, \texttt{b}, \texttt{c}, \texttt{d}, \texttt{e}
    can be included into the solution by this rule. As there is no constraint on the
    result of this aggregate, any subset of this set of atoms can be included into
    the solution.
\end{example}

The statement in the Example~\ref{exp:input_space} allows us to easily define the input space
of logic program.

\subsubsection{Show statements}

Often only a part of a solution is needed, while most of the solution can be hidden.
For this, Clingo does support show statement. Such statement takes one of following forms:
\begin{center}
    \begin{tabular}{l}
        \texttt{\#show $p$/$n$.}\\
        \texttt{\#show $t$:$L_1$, \ldots, $L_n$.}\\
        \texttt{\#show.}
    \end{tabular}
\end{center}

In the first form, atoms with predicate symbol $p$ with exactly $n$ parameters
are shown on the output. If at least one such atom is included in the solution,
only atoms specified by \texttt{\#show} statements are shown.
A predicate symbol and its classical negation is taken as two distinct symbols.

The second form allows for showing terms if some literals are consistent with the solution.
For a term $t$ and literals $L_1, \ldots, L_n$, term $t$ is shown on the output
if all literals $L_1, \ldots, L_n$ are consistent with the solution.
This form does not hide any atoms from the solution.

Finally, the third form specifies, that no atom other than these specified by
show statements in the first form should be shown on the output.
