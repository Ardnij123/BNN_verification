\subsection{Gringo}

Gringo~\cite{GebserKKS17} is software that grounds logic program in Clingo language
and translates it into the Aspif format that is readable by logic program solver Clasp.
Gringo first resolves every rule with variables
into (possibly multiple) variable-free rules and then changes format of
the program into Clasp-readable Aspif. Gringo thus can introduce new atoms
that were not obvious from the Clingo program.
\newcommand{\ms}{\texttt{ }}

\subsection{Clasp}\label{sec:clasp}

Clasp is the solver of the Clingo framework. It takes logic program in the Aspif
format to search for answer sets of a grounded logic program.
The solver approaches the inference using the unit propagation
of nogoods~\cite{DBLP:journals/ai/GebserKS12}.

For the inference of answer sets of some logic program,
Clasp uses backpropagation over a tree of partial solutions.
Each time it sees a partial solution that is not an answer set
of the program, it derives new constraints on the solution which it
propagates up the tree of partial solutions. This allows Clasp
to efficiently prune large branches of partial solutions.

For model (answer sets) counting, Clasp uses a method
called model enumeration. To determine that a set is an answer set,
it needs to evaluate the exact set.
This means, that the time complexity
of finding all answer sets is always at least linear with respect to the number of answer sets.
