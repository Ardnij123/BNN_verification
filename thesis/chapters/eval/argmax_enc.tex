\subsection{Evaluation of argmax computation}

For the argmax layer, I have constructed 6 encodings,%
~\Cref{enc:argmax_max,enc:argmax_potential_pm1,enc:argmax_potential_01,enc:argmax_variable,enc:argmax_direct_eq,enc:argmax_direct}.

\begin{table}[p]
    \Crefname{encoding}{Enc.}{}
    \begin{center}
        \makebox[\textwidth][c]{
\begin{tabular}{ l l | r r | r r | r r }  % chktex 44
    \toprule{}%
    & & \multicolumn{2}{c |}{M1} & \multicolumn{2}{c |}{M2} & \multicolumn{2}{c}{M7}\\
    & & I0 & I7 & I0 & I7 & I0 & I7 \\ \midrule
$d=0$ & \Cref{enc:argmax_max}           &   3.216 &   3.126 &   0.758 &   0.779 &   0.201 &   0.194 \\
      & \Cref{enc:argmax_potential_pm1} &   3.226 &   3.197 &   0.828 &   0.804 &   0.185 &   0.176 \\
      & \Cref{enc:argmax_potential_01}  &   0.766 &   0.749 &   0.202 &   0.196 &   0.087 &   0.082 \\
      & \Cref{enc:argmax_variable}      &   0.857 &   0.864 &   0.235 &   0.229 &   0.095 &   0.091 \\
      & \Cref{enc:argmax_direct_eq}     &   0.112 &   0.111 &   0.057 &   0.057 &   0.065 &   0.062 \\
      & \Cref{enc:argmax_direct}        &   0.111 &   0.110 &   0.056 &   0.056 &   0.064 &   0.062 \\\midrule
$d=1$ & \Cref{enc:argmax_max}           &   5.050 &   7.751 &   0.965 &   1.510 &   0.556 &   0.982 \\
      & \Cref{enc:argmax_potential_pm1} &   5.005 &   5.977 &   1.115 &   1.303 &   0.548 &   0.833 \\
      & \Cref{enc:argmax_potential_01}  &   1.463 &   2.064 &   0.340 &   0.452 &   0.419 &   0.738 \\
      & \Cref{enc:argmax_variable}      &   1.568 &   2.164 &   0.369 &   0.495 &   0.475 &   0.724 \\
      & \Cref{enc:argmax_direct_eq}     &   0.294 &   0.549 &   0.117 &   0.181 &   0.375 &   0.619 \\
      & \Cref{enc:argmax_direct}        &   0.294 &   0.544 &   0.117 &   0.184 &   0.344 &   0.634 \\\midrule
$d=2$ & \Cref{enc:argmax_max}           &  51.059 &  18.823 &  12.264 &   8.789 &   3.285 &   2.899 \\
      & \Cref{enc:argmax_potential_pm1} &  18.904 &  16.342 &   5.806 &   4.794 &   2.436 &   2.308 \\
      & \Cref{enc:argmax_potential_01}  &  11.547 &   7.927 &   3.873 &   2.713 &   2.499 &   2.151 \\
      & \Cref{enc:argmax_variable}      &  11.007 &   7.408 &   3.712 &   2.931 &   2.186 &   2.068 \\
      & \Cref{enc:argmax_direct_eq}     &   2.991 &   2.640 &   1.380 &   1.117 &   2.030 &   2.044 \\
      & \Cref{enc:argmax_direct}        &   3.346 &   2.659 &   1.390 &   1.065 &   2.259 &   1.755 \\\midrule
$d=3$ & \Cref{enc:argmax_max}           &     --- &     --- & 199.465 & 133.432 &  75.185 &  74.186 \\
      & \Cref{enc:argmax_potential_pm1} &     --- &     --- & 163.086 & 104.734 &  57.775 &  51.720 \\
      & \Cref{enc:argmax_potential_01}  & 191.495 & 184.556 & 118.226 &  80.181 &  47.022 &  50.710 \\
      & \Cref{enc:argmax_variable}      & 185.261 & 175.619 & 110.676 &  79.620 &  47.209 &  45.417 \\
      & \Cref{enc:argmax_direct_eq}     &  73.132 &  53.837 &  33.121 &  27.664 &  45.230 &  41.119 \\
      & \Cref{enc:argmax_direct}        &  70.076 &  55.665 &  34.035 &  25.567 &  43.001 &  40.295 \\\bottomrule
\end{tabular}}
    \end{center}
    \caption{Computation time for models M1, M2 and M7 on $HD(\text{I0}, d)$ and $HD(\text{I7}, d)$
    for $d\in \{0, 1, 2, 3\}$ and different implementations of argmax layer.\\
    Values in the table were obtained using methodology described in~\Cref{sec:consistency}
    (avg.\ of 3 best of 4). Symbols --- in the table mean that less than 3 of 4 evaluations finished
    before timeout. Timeout was set to 300\,s.
    }\label{tab:eval_argmax_hamming}%
\end{table}

\subsubsection{Over hamming distance}

I have meassured the time to evaluate robustness over input regions
based of hamming distance $HD(\text{I0}, d)$ and $HD(\text{I7}, d)$
for $d\in\{0,1,2,3\}$ on models M1, M2 and M7. Results are in%
~\Cref{tab:eval_argmax_hamming}.

Encodings may be split into 3 groups by the computation time.
\Cref{enc:argmax_max,enc:argmax_potential_pm1}, \Cref{enc:argmax_potential_01,enc:argmax_variable}
and \Cref{enc:argmax_direct_eq,enc:argmax_direct}.

\Cref{enc:argmax_max,enc:argmax_potential_pm1} result in a slow grounded program.
These encodings use predicate symbol \texttt{potential}, with inner potential
computed by sum aggregate over both positive and negative literals of symbol
\texttt{on} according to the default negation. Grounder Gringo does not see
that either a positive or negative version of literal must be true at a time.
The grounded program contains for each perceptron atoms of symbol \texttt{potential}
with values $b-m$ to $b+m$ where $b$ is bias of the perceptron and $m$ lenght of
input of the perceptron, while only values that give the same remainder after division
by 2 as $b+m$ are feasible.

In~\Cref{enc:argmax_max}, atoms of symbol \texttt{potential} are then used in
max aggregate. When grounding this type of aggregate, Gringo generally orders
its terms and for each of them adds a rule that asserts the aggregate to be
equal to the largest term of those whose condition is consistent with the answer set.
As there are many different atoms of symbol \texttt{potential}, this then results
in a large set of rules.

In case of~\Cref{enc:argmax_potential_pm1}, the part dependent on atoms of symbol
\texttt{potential} is the constraint on \texttt{output}. The constraint ground into
constant number of rules for each pair of \texttt{potential} atoms of different outputs.

Moving to~\Cref{enc:argmax_potential_01}, the number of atoms of symbol
\texttt{potential} halves. This results in quartering of the number of rules
and a significant speedup. \Cref{enc:argmax_variable} is similiar.
Resulting logic program still has atoms for individual values of inner potential,
however now it has them for each pair of distinct outputs individually.

Finally, \Cref{enc:argmax_direct_eq,enc:argmax_direct} hold only a single
constraint based of aggregate for every pair of outputs. This largely
decreases the size of the grounded program as all but these have been storing
some form of all possible inner potential values. These two are grounding into a rule
directly computing the inner potential.

\begin{table}[h!]
    \Crefname{encoding}{Enc.}{}
    \begin{center}
        \makebox[\textwidth][c]{
\begin{tabular}{ l l | r r | r r | r r }  % chktex 44
    \toprule{}%
    & & \multicolumn{2}{c |}{M1} & \multicolumn{2}{c |}{M2} & \multicolumn{2}{c}{M7}\\
    & & I0 & I7 & I0 & I7 & I0 & I7 \\ \midrule
$F=0$ & \Cref{enc:argmax_max}           &   3.147 &   3.030 &   0.752 &   0.762 &   0.193 &   0.183 \\
      & \Cref{enc:argmax_potential_pm1} &   3.012 &   2.913 &   0.785 &   0.760 &   0.173 &   0.164 \\
      & \Cref{enc:argmax_potential_01}  &   0.689 &   0.667 &   0.183 &   0.177 &   0.082 &   0.076 \\
      & \Cref{enc:argmax_variable}      &   0.787 &   0.766 &   0.214 &   0.209 &   0.091 &   0.085 \\
      & \Cref{enc:argmax_direct_eq}     &   0.106 &   0.105 &   0.054 &   0.054 &   0.060 &   0.058 \\
      & \Cref{enc:argmax_direct}        &   0.105 &   0.104 &   0.054 &   0.053 &   0.060 &   0.058 \\\midrule
$F=8$ & \Cref{enc:argmax_max}           &   3.928 &   4.220 &   0.932 &   0.931 &   0.332 &   0.302 \\
      & \Cref{enc:argmax_potential_pm1} &  11.557 &  12.393 &   1.772 &   1.777 &   0.256 &   0.246 \\
      & \Cref{enc:argmax_potential_01}  &   1.320 &   1.399 &   0.281 &   0.288 &   0.130 &   0.139 \\
      & \Cref{enc:argmax_variable}      &   1.422 &   1.502 &   0.316 &   0.326 &   0.136 &   0.173 \\
      & \Cref{enc:argmax_direct_eq}     &   0.120 &   0.119 &   0.063 &   0.070 &   0.118 &   0.114 \\
      & \Cref{enc:argmax_direct}        &   0.119 &   0.118 &   0.062 &   0.066 &   0.108 &   0.095 \\\midrule
$F=16$& \Cref{enc:argmax_max}           &  29.282 &  30.253 &   5.244 &   6.693 &   4.212 &   4.926 \\
      & \Cref{enc:argmax_potential_pm1} &  30.858 &  26.547 &   7.852 &   6.070 &   3.497 &   3.960 \\
      & \Cref{enc:argmax_potential_01}  &   9.125 &   7.092 &   2.766 &   2.151 &   1.895 &   2.125 \\
      & \Cref{enc:argmax_variable}      &   8.877 &   7.003 &   3.022 &   2.305 &   1.947 &   2.224 \\
      & \Cref{enc:argmax_direct_eq}     &   1.316 &   1.589 &   0.464 &   0.746 &   1.627 &   1.663 \\
      & \Cref{enc:argmax_direct}        &   1.321 &   1.560 &   0.477 &   0.714 &   1.533 &   1.633 \\\bottomrule
\end{tabular}}
    \end{center}
    \caption{Computation time for models M1, M2 and M7 on $R(\text{I0}, \{F,\ldots,100\})$ and $HD(\text{I7}, \{F,\ldots,100\})$
    for $F\in \{0, 8, 16\}$ and different implementations of perceptron.
    Values in the table were obtained using methodology described in~\Cref{sec:consistency}
    (avg.\ of 3 best of 4).
    }\label{tab:eval_argmax_fixed_bits}%
\end{table}

\subsubsection{Over fixed bits}

I have meassured the time to evaluate robustness of models M1, M2 and M7
over input regions based on fixed bits $R(\text{I0}, \{F+1,\ldots,100\})$, $R(\text{I7}, \{F+1,\ldots,100\})$,
where $F\in\{0, 8, 16\}$,
that is 0, 8 and 16 free bits at the begining of the input vector.
Results are in~\Cref{tab:eval_argmax_fixed_bits}.

As in the~\Cref{tab:eval_argmax_hamming}, the encodings were separated to 3--4 groups
by the computation time. The differences are similliar. In this evaluation,
\Cref{enc:argmax_direct} is marginally better than \Cref{enc:argmax_direct}.
