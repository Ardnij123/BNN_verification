\chapter{Evaluation}

In this part I discuss the speed of individual encoding components.
For the best combination of components. I also show the capabilities
of implemented framework on multiple BNN models and inputs.

\section{Methodology of evaluation}

For the evaluation of my framework, I have used inputs from MNIST dataset~\cite{mnist2017},
the same as were used for the evaluation of BNNQuanalyst~\cite{10.1145/3563212}.
In the next chapter I will be reffering to instances of inputs corresponding to
numbers 0 to 9 as I0 to I9.
% TODO: pictures

Models for evaluation have been taken from~\cite{10.1145/3563212}.
The architectures of different models is written in \Cref{table:model_architecture}.
Values in the column \textit{Architecture} correspond to layers sizes. First is
size of the input vector, then follow inner layers and last is size of the output.
Training of BNN models is not part of this thesis and thus is not discussed here.

\begin{table}[H]
\begin{tabular}{l c | l c}  % chktex 44
    \toprule{}%
    Model & Architecture  & Model & Architecture       \\ \midrule
    M1    & 100:100:10    & M7    & 100:50:20:10       \\
    M2    & 100:50:10     & M8    & 16:25:20:10        \\
    M3    & 400:100:10    & M9    & 36:15:10:10        \\
    M4    & 64:10:10      & M10   & 16:64:32:20:10     \\
    M5    & 784:100:10    & M11   & 25:25:25:20:10     \\
    M6    & 100:100:50:10 & M12   & 784:50:50:50:50:10 \\ \bottomrule
\end{tabular}
\caption{Architectures of models}%
    \label{table:model_architecture}
\end{table}

Unless otherwise specified, the framework was evaluated on Dell G3 3579.
For full specifications of both the machine and the system, see~\Cref{app:specs}.

%\section{Evaluation of different solvers}
%I have evaluated BNN models 
% TODO: if time will suffice

