\chapter{Proofs and examples}

\section{Proofs}

\begin{lemma}{The Dirichlet function $D$ is robust on any interval $\langle k, l\rangle$
    where $k, l$ are rational numbers, $k\neq l$.}\label{proof:dirichlet}
    \begin{proof}
        As shown in the Example 3.1.1 of~\cite{HONG200565},
        the Dirichlet function has Lebesgue integral
        on interval $[0, 1]$ with a value equal to $0$.

% Commented lines are proving b in Q <=> ab in Q
%        Lets prove that for every rational number $a\in \QQ\setminus \{0\}$
        Lets prove that for every rational number $a\in \QQ$
        and real number $b\in \RR$
        following statements are equivalent:
        \begin{enumerate}\setlength{\itemsep}{0pt}
            \item $b$ is rational
            \item $a+b$ is rational
%            \item $a\cdot b$ is rational
        \end{enumerate}

%        1. $\implies$ 2., 1. $\implies$ 3.:\\
        1. $\implies$ 2.:\\
        Since both $a$ and $b$ are rational,
        by definition they can be written as a fraction of integers
        \begin{equation*}
            a = \frac{p}{q}, b = \frac{p'}{q'}
        \end{equation*}
%        The sum and product then can be expressed as a fraction of integers,
        The sum can be expressed as a fraction of integers,
%        thus are also rational.
        thus is also rational.
        \begin{equation*}
%            a+b = \frac{pq'+p'q}{qq'}, a\cdot b = \frac{pp'}{qq'}
            a+b = \frac{pq'+p'q}{qq'}
        \end{equation*}

%        2. $\implies$ 1., 3. $\implies$ 1.:\\
        2. $\implies$ 1.:\\
        The $b$ can be written using the rational $a$
%        and the sum or product of $a$ and $b$.
        and the sum of $a$ and $b$.
        \begin{equation*}
%            b = (a+b) + (-a), b = (a\cdot b)\cdot \frac{1}{a}
            b = (a+b) + (-a)
        \end{equation*}
%        Now since both $-a$, $\frac{1}{a}$ and $(a+b)$ or $(a\cdot b)$ are rational,
        Now since both $(-a)$ and $(a+b)$ are rational,
        $b$ was already proven to be rational in the opposite side implication.

%        For $a=0$ also the first and second statements are equivalent trivially.

        Lets show that the integral of Dirichlet function $D$ is equal
        to $0$ on any interval bounded by rational numbers.
        The limits of the definite integral can be transformed by removing $k$
        and adding it to the argument of $D$. As was proven previously,
        for rational number $k$, $D(x+k) = D(x)$.
        \begin{equation*}
            \int_{k}^{l} D(x) dx = \int_{0}^{l-k} D(x+k) dx = \int_{0}^{l-k} D(x) dx
        \end{equation*}
        As the dirichlet function is nonnegative, following is true.
        \begin{equation*}
            0 \leq \int_0^{l-k} D(x) dx \leq \int_0^{\lceil l-k \rceil} D(x) dx
        \end{equation*}
        The right-hand side integral can be split into unit-long parts.
        \begin{equation*}
            \int_0^{\lceil l-k\rceil} D(x) dx
            = \int_0^1 D(x) dx + \int_1^2 D(x) dx + \ldots
                + \int_{\lceil l-k \rceil-1}^{\lceil l-k \rceil} D(x) dx
        \end{equation*}
        Finally, as every unit integral has rational bounds,
        it can be transformed to integral with 0 as lower bound like already shown.
        \begin{equation*}
            \int_u^{u+1} D(x) dx = \int_0^1 D(x+u) dx = \int_0^1 D(x) dx = 0
        \end{equation*}
        \begin{equation*}
            0 \leq \int_{k}^{l} D(x) dx \leq \int_0^{\lceil l-k \rceil} D(x) dx = 0
        \end{equation*}

        To show the robustness of Dirichlet function, the quantitative robustness
        with respect to constant weight function $w_1$
        and identity as the evaluation function can be used.

        \begin{equation*}
            Q_{w_1, id}(\langle k, l\rangle)
            = \frac{\int_{k}^{l} 1\cdot D(x) dx}{\int_{k}^{l} 1} = \frac{0}{l-k} = 0
        \end{equation*}
    \end{proof}
\end{lemma}

\begin{lemma}{Incremental building of the answer set of the basic logic program.}%
    \label{proof:incremental}
    Let $(\Sigma, \Pi)$ be a basic logic program with at most one literal
    in the head of every rule.
    Let
    \begin{equation*}
        S_0 = \emptyset,
    \end{equation*}
    \begin{equation*}
        S_{i+1} = S_{i}\cup \{\varphi \mid \{\varphi\} = \asphead(r), \aspbody^+(r) \subseteq S_i\} \text{ for each } i \geq 0.
    \end{equation*}
    Denote by $S_\infty$ the union of all $S_i$ for $i\geq 0$.
    If it holds that:
    \begin{itemize}
        \item for every $\Sigma$-atom $\varphi$, $S_\infty$ contains at most one
            of literals $\varphi, \neg\varphi$, and
        \item $S_\infty$ is consistent with every constraint of $\Pi$,
    \end{itemize}
    then $S_\infty$ is
    the only answer set of program $(\Sigma, \Pi)$. Else $(\Sigma, \Pi)$ has no answer set.

    For a proof of this lemma, see~\Cref{proof:incremental}.

    \begin{proof}
        $S_\infty$ does contain only (some of) heads of rules from $\Pi$,
        each of these being $\Sigma$-literals. If $S_\infty$ does also not contain
        both $\varphi$ and $\neg\varphi$ for any $\Sigma$-atom $\varphi$,
        $S_\infty$ is a partial interpretation of $\Sigma$.

        By definition $S_\infty$ contains the literal from the head of each rule,
        that has its body consistent with the $S_\infty$. For this the $S_\infty$
        is consistent with every non-constraint rule of the logic program.

        Let $rank: S_\infty \to \NN_0$ be a function that assigns to every literal
        $\varphi \in S_\infty$ the lowest value $n$ such that $\varphi \in S_n$.
        Let there be some answer set $A$ of $(\Sigma, \Pi)$ such that
        $S_\infty\setminus A = \Delta \neq \emptyset$.
        Let $\alpha$ be a literal from $\Delta$ with the lowest $rank$.
        By definition, $rank(\alpha) \geq 1$ as $S_0$ is empty.
        As the literal $\alpha\in S_{rank(\alpha)}$ and $\alpha \not\in S_{rank(\alpha)-1}$,
        either $\alpha$ is in a head of some fact (thus $rank(\alpha)=1$ and $\alpha$
        needs to be in $A$ for it to be consistent),
        or there must be some rule $r\in\Pi$ such that $\aspbody^+(r)$ is not consistent
        with $S_{rank(\alpha)-2}$ and is consistent with $S_{rank(\alpha)-1}$.
        (The rule $r$ is what made the literal $\alpha$ in $S_{rank(\alpha)}$.)
        But $A$ does contain all the literals from $S_{rank(\alpha)-1}$. (Literal $\alpha$
        has the lowest rank of all literals from $S_\infty$ not included in $A$.)
        Thus $A$ is not consistent with $r$ and is not an answer set of $(\Sigma, \Pi)$.
        By contradiction, for every answer set $B$ of $(\Sigma, \Pi)$,
        $S_\infty \subseteq B$. As the answer set is by definition minimal,
        the $S_\infty$ is the only possible answer set of $(\Sigma, \Pi)$.
    \end{proof}
\end{lemma}

\section{Examples}

\begin{example}\label{exp:grounding}
    The ground instantiation of the logic program $(\Sigma_1, \Pi_1)$ defined in the
    \cref{exp:logic_program1} can be found as follows:

    The fact $\aspfact{p(True)}$ does not contain any variable nor function symbol.
    For this, it is ground. The same holds for the fact $\aspfact{q(5, True)}$
    and the rule $\asprule{p(False)}{\aspnot p(True)}$

    To ground the rule $\asprule{q(Y, True)}{q(X, (Y < X))}$,
    this rule should be first substitued for each rule with pair $(X, Y)$ substitued
    for every element from $\sigma_\NN \times \sigma_\NN$, that is every element
    from $\NN_0\times \NN_0$. Then in the second parameter of atom in the body,
    the expression substitued for $(Y < X)$ would be replaced with $True$ or $False$,
    depending on the substitution for $X$ and $Y$. This would lead to the ground
    instantiation $gr(\Sigma_1, \Pi_1)$ being infinite.

    To fight this, we can allow only the instantiation of rule $r$ in which
    only the substitutions of variables in the $\aspbody^+(r)$
    leading to the already existing atoms are allowed.
    After the stable model will be defined,
    it will be easy to see that this constraint does not remove any
    stable models of program, as every atom in it has to also be
    in the head of some rule.

    In this example, we already know that the atom $p(5, True)$ does exist.
    This means we can substitute $X\equiv 5$ and $(Y<X)\equiv True$.
    Also as $(Y<X)\equiv (Y < 5) \equiv True$, to successfully instantiate
    the rule, it has to hold that $Y < 5$. There are no other constraints
    on the rule, thus the rule instantiates into rules:

    \begin{equation*}
        \begin{matrix}
            Y=0 & : & \asprule{q(0, True)}{q(5, True)}\\
            Y=1 & : & \asprule{q(1, True)}{q(5, True)}\\
            Y=2 & : & \asprule{q(2, True)}{q(5, True)}\\
            Y=3 & : & \asprule{q(3, True)}{q(5, True)}\\
            Y=4 & : & \asprule{q(4, True)}{q(5, True)}\\
        \end{matrix}
    \end{equation*}

    But now we have introduced new atoms on predicate $q$. We have to also add
    all instances of the rule, in which these atoms are allowed.
    As the individual atoms only differ in the first term, the instances are
    seen easily. The instances added by the new rules also do not introduce
    new atoms on predicate $q$, the ground instance is thus in this case finite.

    \begin{equation*}
        \begin{matrix}
            X=4, Y=0 & : & \asprule{q(0, True)}{q(4, True)}\\
            X=4, Y=1 & : & \asprule{q(1, True)}{q(4, True)}\\
            X=4, Y=2 & : & \asprule{q(2, True)}{q(4, True)}\\
            X=4, Y=3 & : & \asprule{q(3, True)}{q(4, True)}\\
            X=3, Y=0 & : & \asprule{q(0, True)}{q(3, True)}\\
            X=3, Y=1 & : & \asprule{q(1, True)}{q(3, True)}\\
            X=3, Y=2 & : & \asprule{q(2, True)}{q(3, True)}\\
            X=2, Y=0 & : & \asprule{q(0, True)}{q(2, True)}\\
            X=2, Y=1 & : & \asprule{q(1, True)}{q(2, True)}\\
            X=1, Y=0 & : & \asprule{q(0, True)}{q(1, True)}\\
        \end{matrix}
    \end{equation*}

    The full ground instantiation $gr((\Sigma_1, \Pi_1))$ is thus the program:

    \begin{equation*}
        gr((\Sigma_1, \Pi_1)) =
            \begin{cases}
                \aspfact{p(True)}\\
                \aspfact{q(5, True)}\\
                \asprule{p(False)}{\aspnot p(True)}\\
                \asprule{q(0, True)}{q(5, True)}\\
                \asprule{q(1, True)}{q(5, True)}\\
                \asprule{q(2, True)}{q(5, True)}\\
                \asprule{q(3, True)}{q(5, True)}\\
                \asprule{q(4, True)}{q(5, True)}\\
                \asprule{q(0, True)}{q(4, True)}\\
                \asprule{q(1, True)}{q(4, True)}\\
                \asprule{q(2, True)}{q(4, True)}\\
                \asprule{q(3, True)}{q(4, True)}\\
                \asprule{q(0, True)}{q(3, True)}\\
                \asprule{q(1, True)}{q(3, True)}\\
                \asprule{q(2, True)}{q(3, True)}\\
                \asprule{q(0, True)}{q(2, True)}\\
                \asprule{q(1, True)}{q(2, True)}\\
                \asprule{q(0, True)}{q(1, True)}\\
            \end{cases}
    \end{equation*}
\end{example}

\cref{exp:grounding} well illustrates the need for the logic programs
to be well written. As was shown in the example, due to a single flawed rule,
the size of the ground instantiation of the logic program is quadratic
given the highest number in the first parameter of atoms of the predicate $q$.
