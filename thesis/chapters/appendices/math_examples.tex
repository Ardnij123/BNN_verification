\chapter{Proofs and examples}

\section{Proofs}

\begin{lemma}{The Dirichlet function $D$ is robust on any interval $\langle k, l\rangle$
    where $k, l$ are rational numbers, $k\neq l$.}\label{proof:dirichlet}
    \begin{proof}
        As shown in the Example 3.1.1 of~\cite{HONG200565},
        the Dirichlet function has Lebesgue integral
        on interval $[0, 1]$ with a value equal to $0$.

% Commented lines are proving b in Q <=> ab in Q
%        Lets prove that for every rational number $a\in \QQ\setminus \{0\}$
        Lets prove that for every rational number $a\in \QQ$
        and real number $b\in \RR$
        following statements are equivalent:
        \begin{enumerate}\setlength{\itemsep}{0pt}
            \item $b$ is rational
            \item $a+b$ is rational
%            \item $a\cdot b$ is rational
        \end{enumerate}

%        1. $\implies$ 2., 1. $\implies$ 3.:\\
        1. $\implies$ 2.:\\
        Since both $a$ and $b$ are rational,
        by definition they can be written as a fraction of integers
        \begin{equation*}
            a = \frac{p}{q}, b = \frac{p'}{q'}
        \end{equation*}
%        The sum and product then can be expressed as a fraction of integers,
        The sum can be expressed as a fraction of integers,
%        thus are also rational.
        thus is also rational.
        \begin{equation*}
%            a+b = \frac{pq'+p'q}{qq'}, a\cdot b = \frac{pp'}{qq'}
            a+b = \frac{pq'+p'q}{qq'}
        \end{equation*}

%        2. $\implies$ 1., 3. $\implies$ 1.:\\
        2. $\implies$ 1.:\\
        The $b$ can be written using the rational $a$
%        and the sum or product of $a$ and $b$.
        and the sum of $a$ and $b$.
        \begin{equation*}
%            b = (a+b) + (-a), b = (a\cdot b)\cdot \frac{1}{a}
            b = (a+b) + (-a)
        \end{equation*}
%        Now since both $-a$, $\frac{1}{a}$ and $(a+b)$ or $(a\cdot b)$ are rational,
        Now since both $(-a)$ and $(a+b)$ are rational,
        $b$ was already proven to be rational in the opposite side implication.

%        For $a=0$ also the first and second statements are equivalent trivially.

        Lets show that the integral of Dirichlet function $D$ is equal
        to $0$ on any interval bounded by rational numbers.
        The limits of the definite integral can be transformed by removing $k$
        and adding it to the argument of $D$. As was proven previously,
        for rational number $k$, $D(x+k) = D(x)$.
        \begin{equation*}
            \int_{k}^{l} D(x) dx = \int_{0}^{l-k} D(x+k) dx = \int_{0}^{l-k} D(x) dx
        \end{equation*}
        As the dirichlet function is nonnegative, following is true.
        \begin{equation*}
            0 \leq \int_0^{l-k} D(x) dx \leq \int_0^{\lceil l-k \rceil} D(x) dx
        \end{equation*}
        The right-hand side integral can be split into unit-long parts.
        \begin{equation*}
            \int_0^{\lceil l-k\rceil} D(x) dx
            = \int_0^1 D(x) dx + \int_1^2 D(x) dx + \ldots
                + \int_{\lceil l-k \rceil-1}^{\lceil l-k \rceil} D(x) dx
        \end{equation*}
        Finally, as every unit integral has rational bounds,
        it can be transformed to integral with 0 as lower bound like already shown.
        \begin{equation*}
            \int_u^{u+1} D(x) dx = \int_0^1 D(x+u) dx = \int_0^1 D(x) dx = 0
        \end{equation*}
        \begin{equation*}
            0 \leq \int_{k}^{l} D(x) dx \leq \int_0^{\lceil l-k \rceil} D(x) dx = 0
        \end{equation*}

        To show the robustness of Dirichlet function, the quantitative robustness
        with respect to constant weight function $w_1$
        and identity as the evaluation function can be used.

        \begin{equation*}
            Q_{w_1, id}(\langle k, l\rangle)
            = \frac{\int_{k}^{l} 1\cdot D(x) dx}{\int_{k}^{l} 1} = \frac{0}{l-k} = 0
        \end{equation*}
    \end{proof}
\end{lemma}

