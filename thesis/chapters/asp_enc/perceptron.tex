\section{ASP encoding of BNN}

In the encoding of binarised neural network in Clingo language I use
predicate symbols with meaning described in~\Cref{table:encoding_semantics}.
This table contains atoms with their meaning categorised into ones that
encode the binarised neural network, ones that specify the input region
and ones that are computed while grounding.
Both atoms that define the binarised neural network and that specify the input
region are given to the logic program as facts by the Python encoder.
% TODO: document the usage of classical negation if used

{
\begin{table}[h!]
    \makebox[\textwidth]{%
\begin{tabular}{lLl}
    \toprule{}%
    Symbol & \multicolumn{2}{l}{Semantics}\\
    \midrule{}%
    Encoding of BNN & & \\
    \midrule
    \atom{layer}{L, N} & L & Layer number\\
    & & Input layer has number 0\\
    & & Output layer has the highest number\\
    & N & Number of perceptrons in layer $L$\\
    \atom{weight}{L, M, N, W} & L & Layer\\
    & M & Position of perceptron of layer $L-1$\\
    & N & Position of perceptron in layer $L$\\
    & W & Weight of input $M$ to perceptron $N$ in layer $L$\\
    \atom{bias}{L, N, B} & L & Layer\\
    & N & Position of perceptron in layer $L$\\
    & B & Bias of perceptron $N$ in layer $L$\\
    \atom{outpre}{N, P} & N & Position of output\\
    & P & $N$ has precedence $P$\\
    & & lower precedence mean higher priority\\
    \midrule
    Encoding of input space & & \\
    \midrule
    \atom{input}{N} & & Specification of the base input\\
    & N & Input on the position $N$ of input layer has value 1\\
    \atom{hammdist}{R} & & The input space is based on the hamming distance\\
    & R & Maximal hamming distance from base input is $R$\\
    \atom{inpfix}{N} & & The input space is based on fixed bits\\
    & N & Position $N$ of input is fixed to the value of base input\\
    \midrule
    Computation & & \\
    \midrule
    \atom{output\_layer}{L} & L & Layer number $L$ is the output layer\\
    \atom{potential}{L, N, P} & L & Layer number\\
    & N & Position of perceptron in layer $L$\\
    & P & Perceptron $N$ of layer $L$ has inner potential $P$\\
    \atom{on}{L, N} & L & Layer number\\
    & N & Perceptron on position $N$ of layer $L$ outputs 1\\
    \atom{output}{N} & N & Output on position $N$ of output layer has value 1\\
    \bottomrule
\end{tabular}
    }%
    \caption{Semantics of encoding BNN into Clingo-readable file}%
    \label{table:encoding_semantics}
\end{table}
}

\newgeometry{
    margin=1em,
    noheadfoot, nomarginpar,
}
\begin{landscape}
  \thispagestyle{empty}
  \begin{figure}[]
    \begin{center}
      \scalebox{.8}{
        \begin{tikzpicture}
          \setupnodes{(0,0)}
          \constnode{weight/{1,n_0,n_1,W}}
          \constnode{bias/{1,n_1,B}}
          \constnode{on/{0,n_0}}

          \setupnodes{($(const{on/{0,n_0}}) + (5,0)$)}
          \compnode{potential/{1,n_1,P}}{weight/{1,n_0,n_1,W},bias/{1,n_1,B},on/{0,n_0}}{}

          \setupnodes{($(const{weight/{1,n_0,n_1,W}}) + (10,0)$)}
          \constnode{weight/{2,n_1,n_2,W}}
          \constnode{bias/{2,n_2,B}}
          \compnode{on/{1,n_1}}{}{potential/{1,n_1,P}}

          \setupnodes{($(comp{on/{1,n_1}}) + (5,0)$)}
          \compnode{potential/{2,n_2,P}}{weight/{2,n_1,n_2,W},bias/{2,n_2,B}}{on/{1,n_1}}

          \setupnodes{($(const{weight/{2,n_1,n_2,W}}) + (10,0)$)}
          \constnode{weight/{3,n_2,n_3,W}}
          \constnode{bias/{3,n_3,B}}
          \compnode{on/{2,n_2}}{}{potential/{2,n_2,P}}

          \setupnodes{($(comp{on/{2,n_2}}) + (5,0)$)}
          \compnode{potential/{3,n_3,P}}{weight/{3,n_2,n_3,W},bias/{3,n_3,B}}{on/{2,n_2}}

          \node (end) at ($(comp{potential/{3,n_3,P}}.east) + (3,0)$) {};
          \draw[dotted] (comp{potential/{3,n_3,P}}) -- (end);

          \node (begin) at ($(comp{potential/{2,n_2,P}}) + (-6,-5)$) {};
          \setupnodes{(begin)}
          \constnode{weight/{{d+1},n_d,n_{d+1},W}}
          \constnode{bias/{{d+1},n_{d+1},B}}
          \compnode{on/{d,n_d}}{}{}

          \setupnodes{($(const{bias/{{d+1},n_{d+1},B}}) + (6,0)$)}
          \constnode{outpre/{n_{d+1},N}}
          \compnode{potential/{{d+1},n_{d+1},P}}{weight/{{d+1},n_d,n_{d+1},W},bias/{{d+1},n_{d+1},B}}{on/{d,n_d}}
          \node (continue) at ($(comp{on/{d,n_d}}.west) + (-3,0)$) {};
          \draw[dotted] (continue) -- (comp{on/{d,n_d}});

          \setupnodes{($(comp{potential/{{d+1},n_{d+1},P}}) + (6,0)$)}
          \compnode{output/{N}}{outpre/{n_{d+1},N}}{potential/{{d+1},n_{d+1},P}}
        \end{tikzpicture}
      }
      \caption{Schema of a multi-layer perceptron encoding}%
      \label{fig:encoding_architecture}
    \end{center}
  \end{figure}
\end{landscape}
\restoregeometry{}

The encoding of BNN into Clingo language will be a composition of layers.
Output of each layer is dependent only on the output values, weights and biases
of the previous layer. Schema of inference for this architecture is shown
in~\Cref{fig:encoding_architecture}. The schema contains two types of nodes,
rectangular nodes do contain data that can be seen as an input of the inference
and nodes with rounded corners that are computed in the runtime of the inference.
As can be seen from the schema, the inference is linear, meaning the value of each
layer is dependent only on the input data or data computed in the previous layer.

\subsection{Encoding of a perceptron}

As shown in~\cref{lem:integer_perceptron}, each binarised perceptron can be encoded
using only bias with value from whole numbers. The encoding in Clingo language is
straightforward.

\begin{code}
\begin{lstlisting}[language=Prolog, numbers=none]
potential(L, N, S+B) :-
    S = #sum{
         W,M :     on(L-1, M), weight(L, M, N, W);
        -W,M : not on(L-1, M), weight(L, M, N, W) },
    bias(L, N, B).

on(L,N) :- potential(L, N, P), P >= 0.
\end{lstlisting}%
    \nobreak\captionof{Encoding}{Encoding of perceptron using potential}\label{enc:perc_potential}
\end{code}

This implementation is a direct encoding of a binarised perceptron.
It follows the formula of perceptron as described in the~\cref{lem:integer_perceptron}.
The implementation is however weak as it relies on the use of intermediate
symbol \texttt{potential}. While grounding, this would require to build a literal
for every possible value of inner potential of each perceptron in the network
and consequently in a large ground program as shown in the~\cref{exp:gringo_grounding}.
To fight that, literals of symbol \texttt{on} may be built directly.\label{sec:perc_potential_removal}

\begin{code}
\begin{lstlisting}[language=Prolog, numbers=none]
on(L, N) :-
    -B <= #sum{
         W,M :     on(L-1, M), weight(L, M, N, W);
        -W,M : not on(L-1, M), weight(L, M, N, W) },
    bias(L, N, B).
\end{lstlisting}
    \nobreak\captionof{Encoding}{Direct encoding of perceptron}\label{enc:perc_direct}
\end{code}

% TODO: Proove that this is correct

The last implementation still has a single flaw. It uses both the positive
and negative variant of the literal \texttt{on(L-1, M)} in sense of the default
negation. We can however make a transformation of the BNN such that it eliminates
use of the negative variant.%
\label{sec:01-mapping}

Starting with the expression from~\ref{lem:integer_perceptron}, the $\pm 1$-binarised
input vector $\vec x$ can be substituted by a $\{1,0\}$-binarised input vector
$\vec x_b$:
\[x_{b,i} = \begin{cases}
    1 & x_i = 1\\
    0 & x_i = -1
\end{cases}\]
\[x_i = 2\cdot x_{b,i} - 1\]
\[\xi(\vec x) = b + \sum_{i=1}^k w_i\cdot (2\cdot x_{b,i} - 1)\]
This expression can be further transformed by splitting the sum and eliminating
the multiplication by 2.
\[\xi(\vec x) = b + \sum_{i=1}^k w_i\cdot 2 \cdot x_{b,i} - \sum_{i=1}^k w_i\cdot 1\]
\[\xi(\vec x) = b - \sum_{i=1}^k w_i + 2\sum_{i=1}^k w_i \cdot x_{b,i}\]
\[p^\BB(\vec x) = H \circ \xi (\vec x) = \begin{cases}
    1 & b - \sum_{i=1}^k w_i + 2\sum_{i=1}^k w_i \cdot x_{b,i} \geq 0\\
    -1 & b - \sum_{i=1}^k w_i + 2\sum_{i=1}^k w_i \cdot x_{b,i} < 0
\end{cases}\]
Finally, both cases can be divided by 2. The expression $b-\sum_{i=1}^k w_i\over 2$
in the final form of equality is independent of the input vector.
It can be seen as the new bias, thus integer part of this expression can take its
place similiarly to~\cref{lem:integer_perceptron}.
\[p^\BB(\vec x) = H \circ \xi (\vec x) = \begin{cases}
    1 & \floor{{b - \sum_{i=1}^k w_i \over 2}} + \sum_{i=1}^k w_i \cdot x_{b,i} \geq 0\\
    {} & {}\\
    -1 & \floor{{b - \sum_{i=1}^k w_i \over 2}} + \sum_{i=1}^k w_i \cdot x_{b,i} < 0
\end{cases}\]

With weights and biases adjusted in the way described above, the perceptron can be
encoded without the use of default negation:

\begin{code}
\begin{lstlisting}[language=Prolog, numbers=none]
on(L, N) :-
    -B <= #sum{ W,M : on(L-1, M), weight(L, M, N, W) },
    bias(L, N, B).
\end{lstlisting}
    \nobreak\captionof{Encoding}{Direct encoding of perceptron using (0,1)-binarised values}\label{enc:perc_direct_01}
\end{code}

\subsection{Encoding of an Argmax layer}

As shown in~\cref{lem:integer_argmax} and remark that follows it, Argmax layer can
be encoded using the vector of weights, biases and the precedence of individual outputs.
First, literals of predicate symbol \texttt{potential} can be constructed similiarly
to the perceptron, on these \texttt{max} aggregate can be used.

\begin{code}
\begin{lstlisting}[language=Prolog, numbers=none]
potential(D+1, N, S+B) :-
    S = #sum{
         W,M :     on(D, M), weight(D, M, N, W);
        -W,M : not on(D, M), weight(D, M, N, W) },
    bias(D, N, B), output_layer(D+1).

output(N) :-
    (Sum, Precedence, Node) = #max{
        (S, -P, N) : potential(D+1, N, S), outpre(N, P) },
    output_layer(D+1).
\end{lstlisting}
    \nobreak\captionof{Encoding}{Encoding of argmax layer using potential and max aggregate}\label{enc:argmax_max}
\end{code}

While this may work, in the grounding process it will expand into a large ruleset.
The ruleset will contain a rule with \texttt{potential} in head for every
possible value of sum of inputs and bias, for each of these multiple rules
to implement the \texttt{max} aggregate will be created.

The \texttt{potential} can however not be directly substituted for a comparison
in the body of \texttt{output} like in the implementation of perceptron
as the value of inner potential is needed for the comparison of different output values.

To overcome this problem, it can be observed from the other side.
Instead of asking wether the particular output position has the largest inner potential,
all output position with inner potential less than some other can be forbidden from being
the final output. An \texttt{output} literal can then be introduced into the solution
using a rule with count aggregate in its head.

\begin{code}
\begin{lstlisting}[language=Prolog, numbers=none]
potential(D+1, N, S+B) :-
    S = #sum{
         W,M :     on(D, M), weight(D+1, M, N, W);
        -W,M : not on(D, M), weight(D+1, M, N, W) },
    bias(D+1, N, B), output_layer(D+1).

1 { output(1..N) } 1 :- output_layer(D+1), layer(D+1, N).

:- output(N),
    potential(D+1, N, S), potential(D+1, M, T),
    output_layer(D+1),
    N != M, S < T.
:- output(N),
    potential(D+1, N, S), potential(D+1, M, T),
    outpre(N, P), outpre(M, Q),
    output_layer(D+1),
    N != M, S = T, P > Q.
\end{lstlisting}
    \nobreak\captionof{Encoding}{Encoding of argmax layer using potential and constraints on output}\label{enc:argmax_potential_pm1}
\end{code}

Negated literals with predicate symbol \texttt{on} can be eliminated similiarly
to the elimination in the encoding of perceptron (see~\Cref{sec:01-mapping}). The precedence must be however
calculated from the new bias after elimination.
(For the implementation see~\Cref{sec:pyt_argmax}.)%
\label{sec:01-argmax}

\begin{code}
\begin{lstlisting}[language=Prolog, numbers=none]
potential(D+1, N, S+B) :-
    S = #sum{ W,M : on(D, M), weight(D+1, M, N, W) },
    bias(D+1, N, B), output_layer(D+1).

1 { output(1..N) } 1 :- output_layer(D+1), layer(D+1, N).

:- output(N),
    potential(D+1, N, S), potential(D+1, M, T),
    output_layer(D+1),
    N != M, S < T.
:- output(N),
    potential(D+1, N, S), potential(D+1, M, T),
    outpre(N, P), outpre(M, Q),
    output_layer(D+1),
    N != M, S = T, P > Q.
\end{lstlisting}
    \nobreak\captionof{Encoding}{Encoding of argmax layer using potential and constraints on output with $(0,1)$-binarization}\label{enc:argmax_potential_01}
\end{code}

Literals with predicate symbol \texttt{potential} can now be also eliminated
by using \texttt{sum} aggregate directly in the comparison of inner potential values.

\begin{code}
\begin{lstlisting}[language=Prolog, numbers=none]
1 { output(1..N) } 1 :- output_layer(D+1), layer(D+1, N).

:- output(N),
    S = #sum{ W,O : on(D, O), weight(D+1, O, N, W) },
    T = #sum{ W,O : on(D, O), weight(D+1, O, M, W) },
    bias(D+1, N, B), bias(D+1, M, C),
    output_layer(D+1),
    N != M, S + B < T + C.
:- output(N),
    S = #sum{ W,O : on(D, O), weight(D+1, O, N, W) },
    T = #sum{ W,O : on(D, O), weight(D+1, O, M, W) },
    bias(D+1, N, B), bias(D+1, M, C),
    outpre(N, P), outpre(M, Q),
    output_layer(D+1),
    N != M, S + B = T + C, P > Q.
\end{lstlisting}
    \nobreak\captionof{Encoding}{Direct encoding of argmax layer using constraints on output and equality}\label{enc:argmax_variable}
\end{code}

While grounding, the value of an aggregate may be only compared to an integer value.
It may not be compared with value of another aggregate directrly. This results in
a grounding of this logic program being more or less the same as of the last logic program
using potential (in fact, the grounded program is even bigger than before).
To avert this, both aggregates have to be merged into a single one.
\[S + B < T + C\]
\[S - T < C - B\]
The expression is easily transformed for variables representing values of aggregates to be
on one side. As both aggregates are \texttt{sum} aggregates, values of aggregate with variable
\texttt{T} may be multiplied by -1. As the computation of aggregate expression uses
representation as sets, additional value must be added to tuples for summation to distinguish
between values of weights for node \texttt{N} and \texttt{M}.
The same transformation may be done also with the second constraint of the logic program.

\begin{code}
\begin{lstlisting}[language=Prolog, numbers=none]
1 { output(1..N) } 1 :- output_layer(D+1), layer(D+1, N).

:- output(N),
    C - B < #sum{  W,O, 1 : on(D, O), weight(D+1, O, N, W) ;
                  -W,O,-1 : on(D, O), weight(D+1, O, M, W) },
    bias(D+1, N, B), bias(D+1, M, C),
    output_layer(D+1),
    N != M.
:- output(N),
    C - B = #sum{  W,O, 1 : on(D, O), weight(D+1, O, N, W) ;
                  -W,O,-1 : on(D, O), weight(D+1, O, M, W) },
    bias(D+1, N, B), bias(D+1, M, C),
    outpre(N, P), outpre(M, Q),
    output_layer(D+1),
    N != M, P > Q.
\end{lstlisting}
    \nobreak\captionof{Encoding}{Direct encoding of argmax layer using constraints on output and and equality}\label{enc:argmax_direct_eq}
\end{code}

Finally, as described in~\cref{exp:gringo_grounding}, when an equality symbol is used
in an aggregate, while grounding it is transformed into two inequalities.
To further simplify (and shorten) the grounded logic program, symbol \texttt{<=}
may be used instead of \texttt{=} in the second constraint.
If $C-B < S-T$, then the output is not the right one by the first constraint already,
thus it is equivalent to the previous logic program.

\begin{code}
\begin{lstlisting}[language=Prolog, numbers=none]
1 { output(1..N) } 1 :- output_layer(D+1), layer(D+1, N).

:- output(N),
    C - B < #sum{  W,O, 1 : on(D, O), weight(D+1, O, N, W) ;
                  -W,O,-1 : on(D, O), weight(D+1, O, M, W) },
    bias(D+1, N, B), bias(D+1, M, C),
    output_layer(D+1),
    N != M.
:- output(N),
    C - B <= #sum{  W,O, 1 : on(D, O), weight(D+1, O, N, W) ;
                   -W,O,-1 : on(D, O), weight(D+1, O, M, W) },
    bias(D+1, N, B), bias(D+1, M, C),
    outpre(N, P), outpre(M, Q),
    output_layer(D+1),
    N != M, P > Q.
\end{lstlisting}
    \nobreak\captionof{Encoding}{Direct encoding of argmax layer using constraints on output and inequality}\label{enc:argmax_direct}
\end{code}

