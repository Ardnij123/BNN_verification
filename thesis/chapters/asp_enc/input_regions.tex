\subsection{Encoding of input regions}

Similiarly to the final encoding of the argrmax layer, I will encode input region
by first allowing for any input vector using an head aggregate with unspecified boundaries
over literals \texttt{on(0, 1..N)}, on top of this I will then build constraints % chktex 36
defined by the desired input region.

\subsubsection{Input region based on the Hamming distance}

To constrain input such that only inputs with hamming distance at most $r$ are allowed,
a simple aggregate rule can be added into the set of rules.
The rule denies any partial solution with input that contains more than $r$ possitions
differing from the base input to be an answer set.

\begin{code}
\begin{lstlisting}[language=Prolog, numbers=none]
{ on(0, 1..N) } :- layer(0, N).
:- #count{ N :     on(0, N), not input(N);
           N : not on(0, N),     input(N) } > R,
    hammdist(R).
\end{lstlisting}
    \nobreak\captionof{Encoding}{Encoding of input region based on Hamming distance}\label{enc:hamming}
\end{code}

\subsubsection{Input region based on the fixed bits}

Constraining the input on some fixed bits is trivial. To make such constraint,
for each fixed position, one can forbid the base input position having value 1
and real input position not having 1 and vice versa.

\begin{code}
\begin{lstlisting}[language=Prolog, numbers=none]
{ on(0, 1..N) } :- layer(0, N).
:-     on(0, N), not input(N), inpfix(N).
:- not on(0, N),     input(N), inpfix(N).
\end{lstlisting}
    \nobreak\captionof{Encoding}{Encoding of input region based on fixed bits using constraints on input}\label{enc:fixed_bits}
\end{code}

The second constraint can be substitued for a rule with \texttt{on(0,N)} % chktex 36
in its head and not in its body.

\begin{code}
\begin{lstlisting}[language=Prolog, numbers=none]
{ on(0, 1..N) } :- layer(0, N).
:- on(0, N), not input(N), inpfix(N).
on(0, N) :- input(N), inpfix(N).
\end{lstlisting}
    \nobreak\captionof{Encoding}{Encoding of input region based on fixed bits using rule and constraint}\label{enc:fixed_input}
\end{code}

Finally, instead of allowing for any input to have either value in the start,
one can encode the possibility for another value only for the input positions
that are not fixed. This may allow for further optimizations in the grounding process.

\begin{code}
\begin{lstlisting}[language=Prolog, numbers=none]
on(0, N) :- input(N), inpfix(N).
{ on(0, K) } :- not inpfix(K), layer(0, N), K = 1..N.
\end{lstlisting}
    \nobreak\captionof{Encoding}{Encoding of input region based on fixed bits using rule for fixed bits and choice for unfixed bits}\label{enc:fixed_direct}
\end{code}
