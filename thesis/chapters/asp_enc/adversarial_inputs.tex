\subsection{Encoding of robustness}

As already discussed in~\cref{sec:clasp}, for the inference on a logic program,
Clasp uses model enumeration. This means that it needs to enter every leaf node
of a tree of partial solutions that is a model (answer set) of this logic program.
On the other hand, if part of this tree does not contain any model, it may prune
this whole branch.

The robustness as defined in the~\cref{def:quantitative_robustness}
does not depend on inputs that yield output values for which the evaluation function
$h_p$ does yield 0. By the previous observation it is good to remove all such inputs
from the models of the logic program.

Clingo itself can not directly compute the quantitative robustness of the binarised
neural network. It can however still count the number of its models.
By forbidding the desired output value in the logic program,
only the partial solution that do not output this desired value
(thus break the robustness) are left as models of the logic program.
When using the evaluation function $h$ that assigns $0$ to forbidden output values
and $1$ to all other values and weight function $w(x) = 1$,
the quantitative robustness can be easily computed from the number of found models
and the size of the input region. The size of the input region can in turn
be found out using~\cref{lemma:size_of_hamming,lemma:size_of_fixed}.

\[Q(I) = \frac{\sum_{i\in I} w(i)\cdot \overline{h_p}(F(i))}{\sum_{i\in I} w(i)}
    = \frac{\sum_{i\in I} 1\cdot \overline{h_p}(F(i))}{\sum_{i\in I} 1}\]
As only the models of such logic program yield $\overline{h_p}(F(i)) = 1$,
the upper part can be substitued for the number of models. The lower part corresponds
to the size of the input region.
\[Q(I) = \frac{\#models}{||input\_region||}\]

Forbidden outputs can be specified using constraints on literals
with predicate symbol \texttt{output}.
In case of an evaluation function $\overline{h_p}$ from~\cref{def:strict_robustness}
that assigns $0$ to output value $k$, the constraint would be:
\begin{lstlisting}[language=Prolog, numbers=none, escapeinside={|}{|}]
:- output(|$k$|).
\end{lstlisting}
as the value $k$ does lead to the desired output and is thus forbidden. On the other hand,
to encode $t$-target robustness (see end of~\Cref{sec:t-target_robustness})
following constraint could be used:
\begin{lstlisting}[language=Prolog, numbers=none, escapeinside={|}{|}]
:- not output(|$t$|).
\end{lstlisting}
as only the output value $t$ is not desired.
