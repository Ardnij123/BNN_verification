\section{Definition of Binarised neural network}

\subsection{Deep neural network}

General neural network is a multilayer perceptron. It consists of perceptrons (neurons) in layers.
Layers are further split into the input layer, which does not contain any perceptrons,
the output layer and possibly multiple hidden inner layers.
Using this structure, general neural network implements a function $\RR^m\to\RR^n$,
such that $m$ entities in the input layer hold input values,
and $n$ perceptrons of the output layer create output values.
Input layer does not consist of perceptrons, it only holds input values.
Each perceptron computes its inner potential $\xi$ based on the outputs of
the previous layer, its output is then determined by an activation function $\sigma$.

% Is it only the weighted sum? Strictly for perceptron it is i guess.
In case of dense deep neural networks which are discussed in this thesis,
the inner potential $\xi$ is computed as a weighted sum of outputs of all entities
(input values or perceptrons outputs) of the previous layer.
Many different activation functions $\sigma$ are used in practice, such as
\textit{unit step function}, \textit{logistic sigmoid},
\textit{hyperbolic tangens} or \textit{ReLU}.

Multiple of these perceptrons are then assorted into layers. The input layer does not
consist of perceptrons but is directly composed of inputs. The output layer is
often represented by perceptrons with activation function different from other layers, such as
\textit{argmax} for singe-choice classification and
\textit{softmax} for classification using probability.

Further follows a more formal definition of a deep neural network
corresponding to the one from~\cite{Bishop1995NeuralNF}.

\sectionsep{}

\begin{figure}[h]
    \begin{center}
    \begin{tikzpicture}
        \setupnodes{(0,0)}
        \valuenode{1}
        \valuenode{x_1}
        \valuenode{x_2}
        \node (lastNode) at ($(lastNode.south) + (0,-1)$) {};
        \valuenode{x_m}
        \draw[thick,dotted,shorten <= 1em,shorten >= 1em] (x_2) -- (x_m);

        \setupnodes{($(x_2) + (4,0)$)}
        \percnode{/}{1/b,x_1/w_1,x_2/w_2,x_m/w_m}
        
        \setupnodes{($(p{/}) + (3.5,0)$)}
        \valuenode{y}
        
        \draw (p{/}.east) -- (y.west);
    \end{tikzpicture}
        \caption{Schema of a perceptron}\label{fig:perceptron_schema}
    \end{center}
\end{figure}

\begin{definition}[Perceptron]
Perceptron $p$ is a function from vector of $m$ real numbers to a real number.
The function $p$ is a composition of an inner potential $\xi$
and an activation function $\sigma$. The inner potential $\xi$ is a weighted sum
parametrized by static vector of real numbers
$\vec w$ of size $m$ and real bias $b$.
The activation function $\sigma$ can be instantized by any real-valued function.

To describe a perceptron, it is common to use a schema such as on~\Cref{fig:perceptron_schema}.

\begin{equation*}
	p: \RR^m \to \RR
\end{equation*}
\begin{equation*}
	\xi: \RR^m \to \RR, \,
	\sigma: \RR \to \RR
\end{equation*}
\begin{equation*}
	\xi(\vec x) = b + \sum_{i=1}^m w_i\cdot x_i
\end{equation*}
\begin{equation*}
	p = \sigma \circ \xi
\end{equation*}
\end{definition}

% This is usually referred to simply as the perceptron
% Thus maybe use single layer of perceptron?
\begin{definition}[Single-layer perceptron]
Single-layer perceptron $t$ is a function from vector of $m$ real numbers
to vector of $n$ real numbers where each value in the resulted vector is computed
by a signle perceptron.

\begin{equation*}
    t: \, \RR^{m} \to \RR^{n}
\end{equation*}
\begin{equation*}
    t(\vec x) = (p_{1}(\vec x), p_{2}(\vec x), \ldots, p_{n}(\vec x))
\end{equation*}
\end{definition}

\begin{figure}[h]
    \begin{center}
    \begin{tikzpicture}
        \setupnodes{(0,0)}
        \valuenode{1}
        \valuenode{x_1}
        \valuenode{x_2}
        \node (lastNode) at ($(lastNode.south) + (0,-1)$) {};
        \valuenode{x_m}
        \draw[thick,dotted,shorten <= 1em,shorten >= 1em] (x_2) -- (x_m);
        
        \setupnodes{($(4,-0.5)$)}
        \percnode{/1}{1/,x_1/,x_2/,x_m/}
        \percnode{/2}{1/,x_1/,x_2/,x_m/}
        \node (lastNode) at ($(lastNode.south) + (0,-1)$) {};
        \percnode{/n}{1/,x_1/,x_2/,x_m/}
        \draw[thick,dotted,shorten <= 1em,shorten >= 1em] (p{/2}) -- (p{/n});

        \setupnodes{($(p{/1}) + (3.5,0)$)}
        \valuenode{y_1}
        \setupnodes{($(p{/2}) + (3.5,0)$)}
        \valuenode{y_2}
        \setupnodes{($(p{/n}) + (3.5,0)$)}
        \valuenode{y_n}
        \draw[thick,dotted,shorten <= 1em,shorten >= 1em] (y_2) -- (y_n);
        
        \foreach \x in {1,2,n}{ \draw (p{/\x}.east) -- (y_\x.west); }
    \end{tikzpicture}
        \caption{Schema of a single-layer perceptron}\label{fig:slp_schema}
    \end{center}
\end{figure}

% This is usually referred to simply as the perceptron
\begin{definition}[Multi-layer perceptron]
Multi-layer perceptron (also called deep neural network)
is a convolution of Single-layer perceptrons.
The last applied layer $t_{d+1}$ is the called output layer,
all other layers are called hidden
layers. The input of a multi-layer perceptron is called input layer,
it does not consist of neurons as other layers, only holds the input vector.

A multi-layer perceptron is often symbolised as a graph of nodes composed
into layers such as on~\Cref{fig:mlp_schema}. Commonly, values $1$
are removed from the schema.

% Is it obvious or should I denote that each layer is funciton
    % t_i: \RR^{n_{i-1}} \to \RR^{n_i}?
\begin{equation*}
    \DNN: \, \RR^{n_0} \to \RR^{n_{d+1}}
\end{equation*}
\begin{equation*}
    \DNN = t_{d+1} \circ t_d \circ \ldots \circ t_1
\end{equation*}
\end{definition}

\begin{figure}[h]
    \begin{center}
    \begin{tikzpicture}
        \setupnodes{(0,0)}
        \valuenode{1}
        \valuenode{x_1}
        \valuenode{x_2}
        \node (lastNode) at ($(lastNode.south) + (0,-1)$) {};
        \valuenode{x_{n_0}}
        \draw[thick,dotted,shorten <= 1em,shorten >= 1em] (x_2) -- (x_{n_0});
        
        \setupnodes{($(1.east) + (3,0)$)}
        \valuenode{1{}}
        \percnode{1/1}{1/,x_1/,x_2/,x_{n_0}/}
        \percnode{1/2}{1/,x_1/,x_2/,x_{n_0}/}
        \node (lastNode) at ($(lastNode.south) + (0,-1)$) {};
        \percnode{1/n_1}{1/,x_1/,x_2/,x_{n_0}/}
        \draw[thick,dotted,shorten <= 1em,shorten >= 1em] (p{1/2}) -- (p{1/n_1});
        \setupnodes{(1{}.east)}
        \valuenode{1}

        \setupnodes{($(1.east) + (3,0)$)}
        \valuenode{1{}}
        \percnode{2/1}{1/,p{1/1}/,p{1/2}/,p{1/n_1}/}
        \percnode{2/2}{1/,p{1/1}/,p{1/2}/,p{1/n_1}/}
        \node (lastNode) at ($(lastNode.south) + (0,-1)$) {};
        \percnode{2/n_2}{1/,p{1/1}/,p{1/2}/,p{1/n_1}/}
        \draw[thick,dotted,shorten <= 1em,shorten >= 1em] (p{2/2}) -- (p{2/n_2});
        \setupnodes{(1{}.east)}
        \valuenode{1}

        \setupnodes{($(p{2/1}.east) + (3.5,0)$)}
        \percnode[dotted]{d+1/1}{1/,p{2/1}/,p{2/2}/,p{2/n_2}/}
        \percnode[dotted]{d+1/2}{1/,p{2/1}/,p{2/2}/,p{2/n_2}/}
        \node (lastNode) at ($(lastNode.south) + (0,-1)$) {};
        \percnode[dotted]{d+1/n_{d+1}}{1/,p{2/1}/,p{2/2}/,p{2/n_2}/}
        \draw[thick,dotted,shorten <= 1em,shorten >= 1em] (p{d+1/2}) -- (p{d+1/n_{d+1}});

        \setupnodes{($(p{d+1/1}.east) + (2,0)$)}
        \valuenode{y_1}
        \setupnodes{($(p{d+1/2}.east) + (2,0)$)}
        \valuenode{y_2}
        \setupnodes{($(p{d+1/n_{d+1}}.east) + (2,0)$)}
        \valuenode{y_{n_{d+1}}}
        \draw[thick,dotted,shorten <= 1em,shorten >= 1em] (y_2) -- (y_{n_{d+1}});

        \draw (p{d+1/1}.east) -- (y_1.west);
        \draw (p{d+1/2}.east) -- (y_2.west);
        \draw (p{d+1/n_{d+1}}.east) -- (y_{n_{d+1}}.west);
    \end{tikzpicture}
        \caption{Schema of a multi-layer perceptron}\label{fig:mlp_schema}
    \end{center}
\end{figure}

\sectionsep{}

\subsection{Binarised neural network}

The computation of general multi-layer perceptron depends on slow multiplication of
floating-point numbers. This motivated the idea of a binarized perceptron%
~\cite{courbariaux2016}.
Such perceptron constrains both the input space $\vec x$ and vector of weights $\vec w$
to vectors of binary values $-1$ and $1$. The activation function is usually
a heavyside step function $H$.
\begin{equation*}
	H(x) = \left\{\begin{array}{ll}
			1 & x \geq 0\\
			-1 & x < 0
		\end{array}\right.
\end{equation*}
Computation of this constrained perceptron is faster than of a general perceptron
as multiplication of two $\pm 1$-binarized values can be done with a single \textit{XNOR}
gate~\cite{courbariaux2016}.

From binarized perceptrons, multi-layer perceptron can be built similiarly
to the general perceptron. In case of the single-choice classification problem,
the output layer consists of weighted sums and \textit{argmax} operator
to classify the largest sum as the output of the multi-layer perceptron.
Such layer is often called after its operator the argmax layer.
% TODO: maybe add some things about the equivalence to normal MLP

Further I provide a formal definition of binarised neural network
using two definition of binarised perceptron.
The first one is useful for the verification task, while the other
(binarised perceptron with batch normalization~\cite{zhang2021bdd4bnn})
is commonly used for training of deep neural networks. I show that
every binarised perceptron with batch normalization
can be transformed into binarised perceptron without batch normalization
vice versa. This shows the equivalence of this definition
to that of~\cite{zhang2021bdd4bnn}.

\sectionsep{}

\begin{definition}[Binarised perceptron]\label{def:binarised_perceptron}
Binarised perceptron $p^\BB$ is a function from vector of $m$ $\pm 1$-binarised values
to a single $\pm 1$-binarised value.
The function $p^\BB$ is a composition of an inner potential $\xi$
and a heavyside step function $H$. The inner potential $\xi$ is weighted sum
parametrized by static vector of $\pm 1$-binarised values
$\vec w$ of size $m$, and real bias $b$.

\begin{equation*}
	p^\BB: \BB^m \to \BB
\end{equation*}
\begin{equation*}
	\xi: \BB^m \to \RR, \,
	H: \RR \to \BB
\end{equation*}
\begin{equation*}
	\xi(\vec x) = b + \sum_{i=1}^m w_i\cdot x_i
\end{equation*}
\begin{equation*}
	p^\BB = H \circ \xi
\end{equation*}
\end{definition}

\begin{definition}[Binarised perceptron with batch normalization]\label{def:perceptron_bn}
Binarised perceptron with batch normalization $\hat{p}^\BB$
is a function from vector of $m$ $\pm 1$-binarised values
to a single $\pm 1$-binarised value.
The function $\hat p^\BB$ is a composition of an inner potential $\xi$,
batch normalization function $\rho$ and a heavyside step function $H$.
The inner potential $\xi$ is weighted sum parametrized by static vector
of $\pm 1$-binarised values $\vec w$ of size $m$, and real bias $b$.
The batch normalization function $\rho$ is a function on real numbers
parametrized by real values $\alpha$, $\gamma$, $\mu$, $\sigma$.

\begin{equation*}
	\hat{p}^\BB: \BB^m \to \BB
\end{equation*}
\begin{equation*}
	\xi: \BB^m \to \RR, \,
	\rho: \RR \to \RR, \,
	H: \RR \to \BB
\end{equation*}
\begin{equation*}
	\xi(\vec x) = b + \sum_{i=1}^m w_i\cdot x_i
\end{equation*}
\begin{equation*}
	\rho(x) = \alpha\cdot \left(\frac{x-\mu}{\sigma}\right) + \gamma
\end{equation*}
\begin{equation*}
	\hat{p}^\BB = H \circ \rho \circ \xi
\end{equation*}
\end{definition}

\begin{lemma}{For every Binarised perceptron $p^\BB$ there is equivalent Binarised perceptron with batch normalization $\hat p^\BB$.}
\begin{proof}
If the parameters of batch normalization function $\rho$ are set to be
$\alpha=\sigma=1$, $\gamma=\mu=0$, function $\rho$ is identity function.
With parameters of inner potential $\xi$ unchanged, the following holds
\begin{equation*}
	p^\BB = H \circ \xi = H \circ \text{id} \circ \xi = H\circ\rho\circ\xi = \hat{p}^\BB
\end{equation*}
\end{proof}
\end{lemma}

\begin{lemma}For every Binarised perceptron with batch normalization $\hat p^\BB$ there is equivalent Binarised perceptron $p^\BB$.\label{lem:batch_perceptron}
\begin{proof}
The idea behind this construction comes from~\cite{zhang2021bdd4bnn}.

The value of $\hat{p}^\BB(\vec x)$ is only determined by the sign of expression
$(\rho \circ \xi)(\vec x)$. Lets thus analyse the inequality
$(\rho\circ\xi)(\vec x) \geq 0$.
\begin{equation*}
	(\rho \circ \xi)(\vec x) = \alpha\cdot \left(\frac{
		b + \sum_{i=1}^k w_i\cdot x_i
	-\mu}{\sigma}\right) + \gamma
\end{equation*}

If $\alpha = 0$, then the expression $\rho\circ\xi$ is a constant function. In that case
the perceptron is equivalent to the one with the value of its bias higher than the length
of input vector for positive constant perceptron or with negative bias with its absolute
value lower than the length of input vector for negative constant perceptron.

By relaxation of the rules for binarised perceptron to allow for value zero in weights,
both of these constant perceptrons may be replaced by a perceptron with zero-valued
weights and nonnegative resp.\ negative bias for positive and negative case.

If $\alpha \neq 0$, the expression can be divided by the term $\frac{\alpha}{\sigma}$.
In the case of this term being negative, the inequality switches and has to be corrected
by further multiplying by $-1$.
\begin{equation*}
	(\rho \circ \xi)(\vec x) \cdot \frac{\sigma}{\alpha} =
	b + \sum_{i=1}^m w_i\cdot x_i -\mu + \frac{\sigma\cdot \gamma}{\alpha}
\end{equation*}
\begin{equation*}
	(\rho \circ \xi)(\vec x) \cdot \frac{\sigma}{\alpha} =
	(b -\mu + \frac{\sigma\cdot \gamma}{\alpha}) + \sum_{i=1}^m w_i\cdot x_i
\end{equation*}
\begin{equation*}
	\frac{\alpha}{\sigma} > 0 :
		(\rho\circ\xi)(\vec x)\geq 0 \iff 
		(b -\mu + \frac{\sigma\cdot \gamma}{\alpha}) + \sum_{i=1}^m w_i\cdot x_i \geq 0
\end{equation*}
\begin{equation*}
    b' = (b -\mu + \frac{\sigma\cdot \gamma}{\alpha}), \vec w' = \vec w
\end{equation*}
\begin{equation*}
	\frac{\alpha}{\sigma} < 0 :
		(\rho\circ\xi)(\vec x)\geq 0 \iff 
		(b -\mu + \frac{\sigma\cdot \gamma}{\alpha}) + \sum_{i=1}^m w_i\cdot x_i \leq 0
\end{equation*}
\begin{equation*}
	\iff (-b +\mu - \frac{\sigma\cdot \gamma}{\alpha}) + \sum_{i=1}^m -w_i\cdot x_i \geq 0
\end{equation*}
\begin{equation*}
    b' = (-b +\mu - \frac{\sigma\cdot \gamma}{\alpha}), \vec w' = -\vec w
\end{equation*}

As can be seen, both cases of $\alpha\over\sigma$ being positive or negative
result in new real bias value $b'$ and vector of weights $\vec w'$.
These values may then be used as a bias and weights of a binarized perceptron.
    \begin{equation*}
        (\rho \circ \xi)(\vec x) = \xi'(\vec x) = b' + \sum_{i=1}^m w'_i \cdot x_i
    \end{equation*}
\end{proof}
\end{lemma}

\begin{remark}
The proof of~\Cref{lem:batch_perceptron} is constructive and is used for encoding
of the quantitative verification problem as ASP problem.
\end{remark}

\begin{remark}
As $\sum_{i=1}^m w_i\cdot x_i$ is always a whole number, bottom whole part of bias
$\lfloor b \rfloor$ can be used in place of bias in inner layers of BNN.\@
In the output layer however the fractional part can make difference when choosing the
maximal input.
\end{remark}

% This is usually referred to simply as the perceptron
\begin{definition}[Binarised single-layer perceptron]
Binarised single-layer perceptron is a function $t^\BB$ from vector of $m$ $\pm 1$-binarised numbers
to vector of $n$ $\pm 1$-binarised numbers where each value in the result vector is computed
by a signle perceptron.

\begin{equation*}
    t^\BB: \, \BB^{m} \to \BB^{n}
\end{equation*}
\begin{equation*}
    t^\BB(\vec x) = (p^\BB_{1}(\vec x), p^\BB_{2}(\vec x), \ldots, p^\BB_{n}(\vec x))
\end{equation*}
\end{definition}

% TODO: this does not read well
\begin{definition}[Argmax layer]\label{def:argmax}
Argmax layer is a function $t^{am}$ which returns the mask of maximal value after
the weighted sum. This mask has form of a one-hot vector, where only the first
position with the maximal value after the weighted sum is assigned value $1$,
all the other positions are assigned value $0$.

\begin{equation*}
	t^{am}: \, \BB^{m} \to {\{0,1\}}^{n}
\end{equation*}
\begin{equation*}
	t^{am}(\vec x) = y,\, y_k = 1 \iff k = \argmax_{i=1}^n(\xi_i(\vec x))
\end{equation*}
\end{definition}

\begin{definition}[Binarised multi-layer perceptron]
Binarised multi-layer perceptron (also called binarised neural network)
is a convolution of binarised single-layer perceptrons.
The last applied layer $t^{am}$ is called the output layer and takes form of argmax layer,
all other layers are called hidden (or inner)
layers. The input of a binarised multi-layer perceptron is called the input layer.

% Is it obvious or should I denote that each layer is funciton
    % t_i: \RR^{n_{i-1}} \to \RR^{n_i}?
\begin{equation*}
    \BNN: \, \BB^{n_0} \to {\{0, 1\}}^{n_{d+1}}
\end{equation*}
\begin{equation*}
    \BNN = t^{am} \circ t_d^\BB \circ \ldots \circ t_1^\BB
\end{equation*}
\end{definition}

% \subsection{Examples of Binarised neural network}

% TODO: Examples of binarised neural networks
% Maybe some logic functions?

\sectionsep{}

% Notes on training of BNNs?
    % Training using binarization following the training of normal NN?
