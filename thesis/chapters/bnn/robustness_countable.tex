\section{Robustness of Binarised neural network}

There are two types of robustness problems. The qualitative robustness
is a problem to determine wether the neural network gives the same (true)
output for all inputs in some input region. The quantitative rubustness
on the other hand determines the part of this input region, which has
the same output as some chosen input of this region.

Further I provide formal definition of robustness on functions in general.
While I provide multiple types of robustness, in the implementation
I will be using only the quantitative robustness for discrete input region
(\Cref{def:quantitative_robustness}) with constant weight $w(i) = 1$
and strict evaluation function $\overline{h_p}$ (\Cref{def:strict_robustness})
due to limitations of Clingo framework.
% TODO: Is this still true?

\subsection{Definition of robustness}

% TODO: Maybe add something like diferentiability?
\begin{definition}[Quantitative robustness of function]\label{def:quantitative_robustness}
    Let $F$ be a function, $F: P \to Q$.
    Let $I$ be an \textit{input region} of $F$, that is $I \subseteq P$.
    Let $w$ be a function, $w: P\to \RR$, $\forall x\in I: w(x) > 0$, called \textit{weight function}.
    Let $h_p$ be a function for some \textit{base input} $p\in P$, $h_p: Q\to \RR$, called \textit{evaluation
    function}, that satisfies
    \begin{equation*}
        h_p(F(p)) = 0
    \end{equation*}
    \begin{equation*}
        \forall x\in Q .\, 0 \leq \, h_p(x) \leq 1 
    \end{equation*}

    \noindent
    Let $Q_{d, h_p}(I)$ be equal to
    \begin{equation*}
        Q_{w, h_p}(I) = \frac{\int_I w(i)\cdot h_p(F(i)) \, di}{\int_I w(i) \, di}
        \hspace{1em} \text{or} \hspace{1em}
        Q_{w, h_p}(I) = \frac{\sum_{i\in I} w(i)\cdot h_p(F(i))}{\sum_{i\in I} w(i)}
    \end{equation*}
    if the input region is non-discrete or discrete respectively.
    Then $Q_{w, h_p}(I)$ is called the \textit{quantitative robustness} of function $F$
    with a weight function $w$ and evaluation function $h_p$ on input region $I$.

    For empty input region $I = \emptyset$
    or non-discrete input regions $I$ such that $\int_I w(i) di = 0$,
    quantitative robustness is defined to be $0$.
\end{definition}

The quantitative robustness is an average of values aquired by
aplying the evaluation function $h_p$ on the input region
weighted by the weight function $w$. \textit{Function highly robust on some
input region will have low value of quantitative robustness for this
region. The quantitative robustness corresponds to the portion of
inputs being adversarial on this region.}

The weight function can be used to make part of the inputs more prominent.
For instance, by the use of weight function such as $w(x) = \frac{1}{1+||p-x||}$,
inputs closer to the base input will have higher weight.

% Example

The simplest example of the weight function is a constant function $w_1(x) = 1$.
With constant function, every input is assigned the same weight,
which leads to the quantitative robustness being an average
of evaluation function $h_p$ applied over all inputs from the input region.

More complex weight functions can be used to give some areas of the input region
higher priority. Such weight function may reflect requirement for the robustness
closer to the base input $p$.

The evaluation function can be used to encode dissimiliarity of an input
from the base input $p$. That could prove useful in case of external metric.
For instance, if evaluating function $F$ on real numbers against some
other function $G$, following metric might be used:

\[h(x) = \begin{cases}
    0 & |F(x) - G(x)| < 1\\
    \log (1/|F(x)-G(x)|) & \text{else}
\end{cases}\]

To further familiarize with the notation of quantitative robustness,
let me here show you a few properties of this notation.

\begin{lemma}{For input region consisting only of the base input $I=\{p\}$,
    the quantitative rubustness is equal to 0}
    \begin{proof}
        \begin{equation*}
            Q_{w, h_p}(\{p\}) = \frac{w(p)\cdot h_p(p)}{w(p)} = \frac{w(p)}{w(p)}\cdot 0 = 0
        \end{equation*}
    \end{proof}
\end{lemma}

\begin{lemma}{Quantitative robustness is between 0 and 1 for every possible combination
    of functions and input regions.}
    \begin{proof}
        Let me first show that the quantitative robustness is nonnegative.
        Both $w$ and $h_p$ are nonnegative on $P$ and $Q$ respectively,
        thus $w(x)\cdot h_p(y)$ is nonnegative for all $(x, y)\in P\times Q$.
        For that also values of integrals and summations over both $w(x)$
        and $w(x)\cdot h_p(y)$ are nonnegative.
        The fraction of two nonnegative values is nonnegative
        and the quantitative robustness is nonnegative.

        By definition holds $0 \leq h_p(y) \leq 1$, thus also $0 \leq h_p(F(x)) \leq 1$.
        As $w(x) \geq 0$, 
        \begin{equation*}
            0 = w(x) \cdot 0 \leq w(x) \cdot h_p(F(x)) \leq w(x)\cdot 1
        \end{equation*}
        Again, integration or summation over any subset $I \subseteq P$
        can be applied onto the latter two expressions.
        \begin{equation*}
            0 \leq \int_{I} w(i) \cdot h_p(F(i))\, di \leq \int_{I} w(i)\, di
        \end{equation*}
        \begin{equation*}
            0 \leq \sum_{i\in I} w(i) \cdot h_p(F(i)) \leq \sum_{i\in I} w(i)
        \end{equation*}
        If the right-hand side term is equal to 0,
        the statement holds by definition of qualitative robustness.
        Else by division by the right-hand side term (nonnegative), following holds
        \begin{equation*}
            0 \leq \frac{\int_{I} w(i) \cdot h_p(F(i))\, di}{\int_{I} w(i)\, di} \leq 1
        \end{equation*}
        \begin{equation*}
            0 \leq \frac{\sum_{i\in I} w(i) \cdot h_p(F(i))}{\sum_{i\in I} w(i)} \leq 1
        \end{equation*}
    \end{proof}
\end{lemma}

While the quantitative robustness does tell us something about how much do
inputs of the input region differ,
the qualitative robustness does only say if there is any wrong output.
This may seem less useful, however it can lead to lower computational expenses.

\begin{definition}[Qualitative robustness of function $F$ on input region $I$]
    Function $F : P\to Q$ is (qualitatively) robust on input region $I\subseteq P$
    with respect to evaluation function $h_p$
    if and only if $Q_{w, h_p}(I) = 0$.
\end{definition}

\begin{lemma}{The property of qualitative robustness is independent of the function $w$.}%
    \label{lemma:robustness_independence_of_weight}
    \begin{proof}
        For empty input region, the lemma holds trivially by definition.

        Otherwise as weight function $w$ is by definition positive,
        all statements of the following chain are equivalent.
        \begin{equation*}
            Q_{w, h_p}(I) = \frac{\sum_{i\in I} w(i)\cdot h_p(F(i))}{\sum_{i\in I} w(i)} = 0
        \end{equation*}
        Since $\sum_{i\in I} w(i) > 0$:
        \begin{equation*}
            \sum_{i\in I} w(i)\cdot h_p(F(i)) = 0
        \end{equation*}
        As both $w$ and $h_p$ are non-negative
        \begin{equation*}
            \forall i\in I. \, w(i)\cdot h_p(F(i)) = 0
        \end{equation*}
        \begin{equation*}
            \forall i\in I. \, w(i) = 0 \vee h_p(F(i)) = 0
        \end{equation*}
        From the definition $\forall x\in P.\, w(x) > 0$
        \begin{equation*}
            \forall i\in I. \, h_p(F(i)) = 0
        \end{equation*}
        This statement is independent of the function $w$,
        thus the lemma holds for the discrete variation.

        % TODO: Is this true?
        % Can it be proven similiarly?
        The non-discrete variant can be proven similarly.
    \end{proof}
\end{lemma}

%\begin{definition}[Qualitative robustness of function $F$ on input region $I$]
%Function $f : P\to Q$ is (qualitatively) robust on input region $I\subseteq P$
%if and only if the function assigns the same element of $Q$
%to every input element of the region $I$.
%
%\begin{equation*}
%f \text{ is robust on } I \iff \forall i, j\in I.\, f(i) = f(j)
%\end{equation*}
%\end{definition}
%
%\begin{definition}{Quantitative robustness of function $f$ on input region $I$
%with the respect to base input $\overline{i}$.}
%Function $f : P \to Q$ is quantitatively robust on input region $I\subseteq P$
%and base input $\overline i\in I$ with a threshold $\epsilon \geq 0$
%if the probability of choosing input from $I$,
%following the uniform distribution,
%to which the function $f$ assigns the same element of $Q$ as to the input $\overline i$
%is at least $\epsilon$.
%
%\begin{equation*}
%f \text{ is robust on } (I, \overline i) \text{ with threshold } \epsilon \iff
%	\frac{\|i \mid i\in I.\, f(i) = f(\overline i)\|}{\|I\|} \geq \epsilon
%\end{equation*}
%\end{definition}

The robustness according to the introduced definition can be contraintuitive
when applied to functions that are not continuous on the input region.
An extreme example of such function is the Dirichlet function $D: \RR\to \RR$.
\begin{equation*}
    D(x) = \left\{
        \begin{matrix}
            1 & \text{ if } x\in \QQ\\
            0 & \text{ if } x\in \RR\setminus \QQ
        \end{matrix}
        \right.
\end{equation*}

\begin{lemma}{The Dirichlet function $D$ is robust on any interval $\langle k, l\rangle$
    where $k, l$ are rational numbers, $k\neq l$.}

    Proof of this lemma is attached as~\Cref{proof:dirichlet}.
\end{lemma}

\noindent\textbf{%
Since this thesis focuses on robustness of discrete input regions,
I will further assume only discrete version of the robustness problem.%
}

\begin{definition}[Strict qualitative robustness of function $F$ on input region $I$]%
    \label{def:strict_robustness}
    Function $F: P\to Q$ is strictly (qualitatively) robust on input region $I\subseteq P$
    if and only if for some $p\in P$ it is robust on this region with respect
    to evaluation function $\overline{h_p}$ defined as follows:

    \begin{equation*}
        \overline{h_p}(q) = \left\{\begin{matrix}
            0 & F(p) = q\\
            1 & F(p) \neq q
        \end{matrix}\right.
    \end{equation*}
\end{definition}

\begin{lemma}{Function $F$ is strictly robust on input region $I$
    if and only if for each two inputs $i, j\in I$, the function $F$ assigns the same value to them
    i.e.\ $F(i)=F(j)$.}
    \begin{proof}
        Proof of equivalence in this lemma is done
        by prooving corresponding implications.

        For empty input region $I=\emptyset$, the statement holds trivially.
        Further in the proof I will always assume nonempty input region.

        First the left-to-right implication.
        Let function $F$ be strictly robust on input region $I$,
        $\overline{h_p}$ being the evaluation function.
        As shown in the proof of \cref{lemma:robustness_independence_of_weight},
        for all instances from input region $i\in I$ holds
        \begin{equation*}
            \overline{h_p}(F(i)) = 0
        \end{equation*}
        By definition of evaluation function $\overline{h_p}$ it also holds
        \begin{equation*}
            F(p) = F(i)
        \end{equation*}
        As this holds for every input $i\in I$, for every two inputs $i, j\in I$:
        \begin{equation*}
            F(i) = F(p) = F(j)
        \end{equation*}

        Now for the opposite implication: Let for every $i, j\in I$, $F(i) = F(j)$.
        Let $p\in I$. As $\overline{h_p}(F(p)) = 0$
        and for every input $i\in I$ it holds that $F(i) = F(p)$,
        $\overline{h_p}(F(i)) = 0$ for every input from the input region $I$.
        The $F$ is thus strictly robust on input region $I$.
    \end{proof}
\end{lemma}

The strict robustness is equivalent to the robustness as defined in~\cite{10.1145/3563212}.
The $t$-target robustness from this article is equivalent to my definition
of robustness with respect to evaluation function $h_t: Q\to \RR$%
\label{sec:t-target_robustness}
\begin{equation*}
    h_t(q) = \left\{\begin{matrix}
        0 & \text{ if } q \neq t\\
        1 & \text{ if } q = t
    \end{matrix}\right.
\end{equation*}
Finally the term $Pr(R(u, \tau))$ is equal to quantitative robustness  % chktex 35
with respect to weight function $w_1$ and evaluation function $h_u$
on input region $R(u, \tau)$.

% TODO: Can this (especially the weight and evaluation function)
%       be implemented using the Clingo?
%       Maybe via the python interface with custom incremental counter?

\subsection{Definition of input regions}

Both the qualitative and the quantitative robustness rely on subsets
of feasible inputs. To be useful when working with binarised neural networks,
they such input regions should operate on vectors.

I provide definition of two classes of input regions, input regions based
on the Hamming distance and input regions with fixed indices.
The input region based on the Hamming distance $R(\vec u, r)$ contains
all input vectors that difer on at most $r$ positions.
The input region with fixed indices (bits) $R(\vec u, I)$ specifies set of indices $I$
on which the input vector may not differ form $I$.
Definition are taken from~\cite{zhang2021bdd4bnn}.

\begin{definition}[Input region based on the Hamming distance]
    For an input $\vec u\in \BB^{n_1}_{\pm 1}$ and an integer $r \geq 0$, let
    $R(\vec u, r) := \{\vec x\in \BB^{n_1}_{\pm 1} \mid HD(\vec x, \vec u) \leq r \}$,
    where $HD(\vec x, \vec u)$ denotes the Hamming distance
    between $\vec x$ and $\vec u$.
\end{definition}

Intuitively, $R(\vec u, r)$ includes input vectors that differ from $\vec u$ on at most
$r$ positions. Examples of such input regions are:
\begin{align*}
    R((1, 1, 1, 1), 1) = \{&(1, 1, 1, 1),\\
    &(-1, 1, 1, 1), (1, -1, 1, 1), (1, 1, -1, 1), (1, 1, 1, -1)\}
\end{align*}
\begin{align*}
    R((-1, 1, -1), 2) = \{&(-1, 1, -1),\\
    &(1, 1, -1), (-1, -1, -1), (-1, 1, 1),\\
    &(-1, -1, 1), (1, 1, 1), (1, -1, -1)\}
\end{align*}

As stated at the beginning of this section, in the implementation part,
I will only use constant weight function $w(i) = 1$. For this specific
weight function and the expression in the divisor of quantitative robustness
holds that:
\[\sum_{i\in I} w(i) = \sum_{i\in I} 1 = ||I||\]
To quickly compute this sum, an explicit formula will be needed.\pagebreak

\begin{lemma}{Let $\vec u$ be a vector from $\BB^{d}_{\pm 1}$.
    Then $||R(\vec u, r)|| = \sum_{i=0}^{\min(r, d)} {d\choose i}$}%
    \label{lemma:size_of_hamming}%
    \begin{proof}
        $R(\vec u, r)$ is union of sets
        \begin{equation*}
            R(\vec u, r) = \bigcup_{i=0}^{r} \{\vec x \mid HD(\vec x, \vec u) = i\}
        \end{equation*}
        These sets are disjoint as elements of each have different number
        of positions changed. The size of each of these sets is equal to
        \begin{equation*}
            ||\{\vec x \mid HD(\vec x, \vec u) = i\}|| = {d\choose i}
        \end{equation*}
        because they consist of $d$ positions, out of which $i$ are choosen to be altered.
        Finally, as for $i$ larger than $d$, ${d\choose i}=0$, the statement holds.
    \end{proof}
\end{lemma}

\begin{definition}[Input region based on fixed bits]
    For an input $\vec u\in \BB^{n_1}_{\pm 1}$ and set of indices $I \subseteq \{1,\ldots,n_1\}$, let
    $R(\vec u, I) := \{\vec x\in \BB^{n_1}_{\pm 1} \mid \forall i\in I.\, x_i = u_i \}$.
\end{definition}

The input region based on fixed bits $R(\vec u, I)$ does specify
the positions which are fixed to the values of the base vector $\vec u$.
Examples of such input regions are:
\begin{align*}
    R((1,1,-1,1), \{1,3,4\}) = \{(1,1,-1,1), (1,-1,-1,1)\}
\end{align*}
\begin{align*}
    R((1,1,1,1), &\{3\}) = \{(1,1,1,1), (-1,1,1,1), (1,-1,1,1), (-1,-1,1,1),\\
    &(1,1,1,-1), (-1,1,1,-1), (1,-1,1,-1), (-1,-1,1,-1)\}\\
\end{align*}

\begin{lemma}{Let $\vec u$ be a vector from $\BB^{d}_{\pm 1}$, $I\subseteq \{1,\ldots,n_1\}$.
    Then $||R(\vec u, I)|| = 2^{d-||I||}$}%
    \label{lemma:size_of_fixed}%
    \begin{proof}
        Each element of $I$ does fix a single position in the input vector.
        The number of variable positions is $d-||I||$, each being instance of
        $+1$ or $-1$. This makes the number of variants $2^{d-||I||}$.
    \end{proof}
\end{lemma}

\begin{remark}
    \cref{lemma:size_of_hamming,lemma:size_of_fixed}
    allow for fast computation of the size of the input region.
\end{remark}

% Definition of the robustness problem
    % Input region variants (hamming, fixed bits)
    % Examples of input regions
