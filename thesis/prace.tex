\documentclass{fithesis}

\thesissetup{
    gender=m,
	author=Jindřich Matuška,
	id=525183,
	type=bc,
	university=mu,
	faculty=fi,
	locale=english,
	department=Department of Computer Systems and Communications,
	date=2024/05/22,
	place=Brno,
	title=Verification of binarised neural networks using ASP,
	keywords={bnn, verification, binarized neural networks
              answer set programming, asp},  % TODO
	abstract={foo bar},  % TODO
	advisor={RNDr. Samuel Pastva, PhD.},
	thanks={foo bar}, % TODO
    bib=bibliography.bib,
}

\usepackage{makecell}
\usepackage{amsmath}
\usepackage{amssymb}

% BNN representation
\newcommand{\BB}{\mathbb{B}}
\newcommand{\RR}{\mathbb{R}}
\newcommand{\NN}{\mathbb{N}}
\newcommand{\BNN}{\mathcal{N}}
\newcommand{\mat}[1]{\mathbf{#1}}
\newcommand{\sign}{\mathrm{sign}}

% ASP representation
\newcommand{\asprule}[2]{#1 \leftarrow #2}
\newcommand{\aspnot}{\mathrm{not}\,}
\newcommand{\asphead}{\mathrm{head}}
\newcommand{\aspbody}{\mathrm{body}}


\begin{document}
\chapter{Introduction}

% - Co řešíme, proč to řešíme
% 	+ NP-úplné problémy
% 		* zde bychom mohli verifikovat ex-post
% 	+ problémy, u kterých neznáme řešení
% 		* případně verifikovat řešení může být samo exponenciálně náročné
% 	+ kritické problémy
% 		* při špatném výsledku nás to může stát hodně životů/peněz
% - aktuální stav verifikace BNN a neuronek obecně (SAT/SMT)
% - práce související s tématem (state of the art)
% - Existuje nějaký framework?
% - jaké jsou aktuální limity pro velikosti verifikovaných sítí
% - možná rozdělit na 2 kapitoly?

Deep neural networks (DNN) are state of the art technology. They are used in many
real-world applications e.g. medicine, self-driving cars, automatic systems,
many of which are critical. % TODO: is this the right meaning?
State of the art deep neural network have up to hundreads of bilions
parameters [https://arxiv.org/pdf/2109.01652.pdf].
That is too many for people to comprehend. We need tools for automation
of verification of these.

While DNNs generally base their computations in floating point numbers,
the computation of output is computationaly hard. For this, quantization,
a new branch of DNN development, where instead of 32 or 64-bits long floating
point numbers, low-bit-width (eg. 4) fixed point numbers are used instead.
This mitigates the high cost of computation and need for high-end devices.
Extreme example of quantization are Binary neural networks (BNN).
These use binary parameters for their computations, resulting in much
cheaper computational price per parameter on specialised devices.

Verification of neural networks is split into two categories. First,
qualitative verification, searches through input space looking for any
adversarial input, e.g. input which results in bad output.
Second, quantitative verification, searches through input space
counting number of adversarial inputs.
State of the art verificators are usually based on satisfability
theories (SAT) or satisfability modulo theories (SMT) paradigm.
% TODO: Is this right?
% TODO: Are there any other sota solutions?
% TODO: Some citation
Even on BNNs these verificators scale only up to % TODO - qualitative verification
and hundreads of parameters for quantitative verification [BNNQuanalyst].
In this thesis I have implemented quantitative BNN validator using ASP paradigm
in framework Clingo from Potassco. I have also evaluated speed of this
validator agnist state of the art implementation [BNNQuanalysist] on mnist
and fashionmnist dataset.

\chapter{Used tools}

\section{Answer set programming}

Answer set programming (ASP) is a form of declarative
programming oriented towards difficult, primarily NP-hard,
search problems. [Lifschitz, Vladimir. Answer set programming. Vol. 3.
Heidelberg: Springer, 2019.]
ASP is particularly suited for solving difficult combinatorial search problems.
[A Glimpse of Answer Set Programming
Christian Anger and Kathrin Konczak and Thomas Linke and Torsten Schaub]
ASP is somewhat closely related to propositional
satisfability checking (SAT)
in sense that the problem is represented as logic program. Difference is in
computational mechanism of finding solution.

\subsection{Syntax and semantics of ASP}

Logic program $\Pi$ in ASP is a set of rules. Each rule $r_i\in \Pi$ has form of
\begin{equation}
    \asprule{\asphead(r_i)}{\aspbody(r_i)}
\end{equation}
$\asphead(r_i)$ is then a single atom $p_0$, while $\aspbody(r_i)$ is a set of zero
or more atoms $\{p_1, \dots, p_m, \aspnot p_{m+1}, \dots \aspnot p_n\}$.
\begin{equation}
    \asprule{p_0}{p_1, \dots, p_m, \aspnot p_{m+1}, \dots, \aspnot p_n}
\end{equation}
Further the $\aspbody(r_i)$ can be split between
$\aspbody^+(r_i) = \{p_1, \dots, p_m\}$
and $\aspbody^-(r_i) = \{p_{m+1}\dots p_n\}$.
If $\forall r_i \in \Pi.\, \aspbody^-(r_i) = \emptyset$, then $\Pi$
is called \textit{basic}.

Semantically rule (2.2) means \textit{If all atoms from $\aspbody^+(r_i)$
are included in answer set and no atom from $\aspbody^-(r_i)$ is in answer set,
then $\asphead(r_i)$ has to be in the set.}

Set of atoms $X$ is closed under a basic program $\Pi$ if for any
$r_i\in \Pi.\,\asphead(r_i)\in X \iff \aspbody(r_i)\subseteq X$.
For general case the concept of \textit{reduct of a program $\Pi$ relative
to a set $X$ of atoms} is needed.
\begin{equation}
    \Pi^X = \{\asphead(r_i)\leftarrow\aspbody^+(r_i)
              \mid r_i\in \Pi,\,\aspbody^-(r_i)\cap X = \emptyset\}
\end{equation}
This program is always basic as it only contains positive atoms.

Let's denote $\mathrm{Cn}(\Pi)$ the minimal set of atoms closed under
a basic program $\Pi$, that is
\begin{equation}
    \mathrm{Cn}(\Pi) \supseteq
    \{\asphead(r_i) \mid r_i\in \Pi, \aspbody(r_i) \subseteq \mathrm{Cn}(\Pi)\}
\end{equation}
% TODO: find article that proves there is only single such minimal set
such set is said to constitute program $\Pi$.
Set $X$ of atoms is said to be answer set of $\Pi$ iff
\begin{equation}
    \mathrm{Cn}(\Pi^X) = X
\end{equation}
In other words, the answer set is a model of $\Pi$ such that its every atom
is grounded in $\Pi$.

Let's illustrate this property on an example.
\begin{equation*}
    \Pi = \{\asprule{p}{p}, \asprule{q}{\aspnot p}\}
\end{equation*}
There are 4 subsets of set of all atoms. First the reduct relative to
the subset is made, on it the calculation of constituting set is made.
If $X = \mathrm{Cn}(\Pi^X)$, then $X$ is one of answer sets.
\begin{center}
    \begin{tabular}{l|l|l}
        $X$ & $\Pi^X$ & $\mathrm{Cn}(\Pi^X)$ \\\hline
        \{\} & $\begin{matrix}\asprule{p}{p}\\\asprule{q}{}\end{matrix}$ & $\{q\}$\\\hline
        \{p\} & $\begin{matrix}\asprule{p}{p}\end{matrix}$ & $\{\}$\\\hline
        \{q\} & $\begin{matrix}\asprule{p}{p}\\\asprule{q}{}\end{matrix}$ & $\{q\}$\\\hline
        \{p, q\} & $\begin{matrix}\asprule{p}{p}\end{matrix}$ & $\{\}$
    \end{tabular}
\end{center}
There is only a single answer set of $\Pi$, that is $\{q\}$. There is
an interesting type of rule in the table, $\asprule{p}{}$. This type of rule
is commonly reffered as \textit{fact} and ensures that atom $p$ is always
included in the answer set.

Another toy example is $\Pi = \{\asprule{p}{\aspnot q}, \asprule{q}{\aspnot p}\}$.
Again, there are 4 subsets of set of all atoms.
\begin{center}
    \begin{tabular}{l|l|l}
        $X$ & $\Pi^X$ & $\mathrm{Cn}(\Pi^X)$ \\\hline
        \{\} & $\begin{matrix}\asprule{p}{}\\\asprule{q}{}\end{matrix}$ & $\{p, q\}$\\\hline
        \{p\} & $\begin{matrix}\asprule{p}{}\end{matrix}$ & $\{p\}$\\\hline
        \{q\} & $\begin{matrix}\asprule{q}{}\end{matrix}$ & $\{q\}$\\\hline
        \{p, q\} & & $\{\}$
    \end{tabular}
\end{center}
This time there are two answer sets of $\Pi$, $\{p\}$ and $\{q\}$.
As we can see, the double not forms a XOR - exactly one of atoms $p, q$ has to
be in the answer set.

\subsection{Language extensions}

% - obecně o ASP
% 	+ solver pro NP-úplné problémy
% - Clingo
% 	+ rozdělení celého frameworku
% 	+ základy použití, kódování problému

\chapter{Můj přínos}

\section{Binary neural network}

I have been working with deterministic binarized neural networks (BNN)
as defined in [Itay Hubara, Matthieu Courbariaux, Daniel Soudry, Ran El-Yaniv, and Yoshua Bengio. 2016. Binarized neural networks. In Proceedings of the Annual Conference on Neural Information Processing Systems. 4107–4115.] and [BNNQuanalys].
For convinience I have rewritten the output as a natural number instead of
one-hot vector.
This article defines deterministic BNN as a convolution of layers.
The input is
encoded as vector $\BB_{\pm 1}^{n_1}$. The BNN is then encoded as tuple of blocks
$(t_1, t_2, \dots, t_d, t_{d+1})$.
\begin{equation*}
    \BNN: \, \BB_{\pm 1}^{n_1} \to \NN_0
\end{equation*}
\begin{equation}
    \BNN = t_{d+1} \circ t_d \circ \dots \circ t_1,
\end{equation}
where for each $i\in \{1, 2, \dots d\}$ $t_i$ is inner block consisting
of LIN layer $t_i^{lin}$, BN layer $t_i^{bn}$ and BIN layer $t_i^{bin}$.

\begin{equation*}
    t_i: \, \BB_{\pm 1}^{n_i} \to \BB_{\pm 1}^{n_{i+1}}
\end{equation*}
\begin{equation}
    t_i = t_i^{bin} \circ t_i^{bn} \circ t_i^{lin}
\end{equation}
Output block $t_{d+1}$ then consists of LIN layer $t_{d+1}^{lin}$ and ARGMAX layer
$t_{d+1}^{am}$.
\begin{equation*}
    t_{d+1}: \, \BB_{\pm 1}^{n_{d+1}} \to \NN_0
\end{equation*}
\begin{equation}
    t_{d+1} = t_{d+1}^{am} \circ t_{d+1}^{lin}
\end{equation}

Each layer is a function with parameters defined in table [table num].

\begin{table}[h]
\makebox[\linewidth]{
\begin{tabular}{llll}
    \hline
    Layer & Function & Parameters & Definition
    \\ \hline
    LIN & $t_i^{lin}: \, \BB_{\pm 1}^{n_i} \to \RR^{n_{i+1}}$ &
    \makecell[l]{
        Weight matrix: $\mat W\in \BB_{\pm 1}^{n_i \times n_{i+1}}$\\
        Bias (row) vector: $\vec b \in \RR^{n_{i+1}}$} &
    \makecell[l]{
        $t_i^{bn}(\vec x) = \vec y$, where $\forall j\in[n_{i+1}]$,\\
        $\vec y_j = \langle \vec x, \mat W_{:, j} \rangle + \vec b_j $
    }
    \\ \hline
    BN & $t_i^{bn}: \, \RR^{n_{i+1}} \to \RR^{n_{i+1}}$ &
    \makecell[l]{
        Weight vector: $\vec \alpha\in \RR^{n_{i+1}}$\\
        Bias vector: $\vec \gamma \in \RR^{n_{i+1}}$\\
        Mean vector: $\vec \mu \in \RR^{n_{i+1}}$\\
        Std. dev. vector: $\vec \sigma \in \RR^{n_{i+1}}$\\
        } &
    \makecell[l]{
        $t_i^{bn}(\vec x) = \vec y$, where $\forall j\in[n_{i+1}]$,\\
        $\vec y_j = \vec \alpha_j
            \cdot \frac{\vec x_j - \vec \mu_j}{\vec \sigma_j}
            + \vec \gamma_j $
    }
    \\ \hline
    BIN & $t_i^{bin}: \, \RR^{n_{i+1}} \to \BB_{\pm 1}^{n_{i+1}}$ &
        \makecell[c]{-} &
    \makecell[l]{
        $t_i^{bin}(\vec x) = \vec y$, where $\forall j\in[n_{i+1}]$,\\
        $\vec y_j = \left\{
            \begin{matrix}
                +1, & \mathrm{if} \, \vec x_j \geq 0; \\
                -1, & \mathrm{otherwise}
            \end{matrix}
            \right.$
    }
    \\ \hline
    ARGMAX & $t_{d+1}^{am}: \, \RR^{n_{d+1}} \to \NN_0$ &
        \makecell[c]{-} &
    \makecell[l]{
        $t_{d+1}^{am}(\vec x) = \arg \max (\vec x )$
        % TODO if multiple argmax is minimal index
    }
    \\ \hline
\end{tabular}
    }
\end{table}

I have transformed the parameters to lower their count and make them integer only.

\begin{equation*}
    t_i(\vec x) = (t_i^{bin} \circ t_i^{bn} \circ t_i^{lin})(\vec x) = y
\end{equation*}
\begin{equation*}
    y_j = \sign_{\pm 1}\left( \alpha_j \cdot \frac{\langle \vec x, \mat W_{:, j}\rangle + b_j - \vec \mu_j}{\vec \sigma_j} - \vec \gamma_j \right)
\end{equation*}

The argument of $\sign_{\pm 1}$ can be further analysed.

\begin{equation*}
    \vec \alpha_j \cdot \frac{\langle \vec x, \mat W_{:, j}\rangle + \vec b_j - \vec \mu_j}{\vec \sigma_j} - \vec \gamma_j \geq 0
\end{equation*}

There are three possible cases of value $\frac{\vec \alpha_j}{\vec \sigma_j}$:

\begin{equation*}
    \begin{matrix}
        \frac{\vec\alpha_j}{\vec\sigma_j} \geq 0: \\
        \langle \vec x, \mat W_{:, j}\rangle + \vec b_j - \vec \mu_j - \frac{\vec \sigma_j}{\vec\alpha_j}\cdot \vec \gamma_j \geq 0\\
        \mat W'_{:, j} = \mat W_{:, j}, \,
        \vec b'_j = \vec b_j - \vec\mu_j - \frac{\vec\sigma_j}{\vec\alpha_j}\cdot \vec\gamma_j\\
        \langle \vec x, \mat W'_{:, j}\rangle + \vec b'_j \geq 0\\
    \end{matrix}
\end{equation*}

\begin{equation*}
    \begin{matrix}
        \frac{\vec\alpha_j}{\vec\sigma_j} \leq 0: \\
        \langle \vec x, \mat W_{:, j}\rangle + \vec b_j - \vec \mu_j - \frac{\vec \sigma_j}{\vec\alpha_j}\cdot \vec \gamma_j \leq 0 \\
        \langle \vec x, -\mat W_{:, j}\rangle - \vec b_j + \vec \mu_j + \frac{\vec \sigma_j}{\vec\alpha_j}\cdot \vec \gamma_j \geq 0\\
        \mat W'_{:, j} = -\mat W_{:, j}, \,
        \vec b'_j = -\vec b_j + \vec\mu_j + \frac{\vec\sigma_j}{\vec\alpha_j}\cdot \vec\gamma_j\\
        \langle \vec x, \mat W'_{:, j}\rangle + \vec b'_j \geq 0\\
    \end{matrix}
\end{equation*}

\begin{equation*}
    \begin{matrix}
        \vec\alpha_j = 0: \\
        - \vec\gamma_j \geq 0\\
        \mat W'_{:, j} = \vec 0, \,
        \vec b'_j = - \vec\gamma_j\\
        \langle \vec x, \mat W'_{:, j}\rangle + \vec b'_j \geq 0\\
    \end{matrix}
\end{equation*}

This shows that each inner layer can be computed using weight matrix and
one bias parameter. However if there were some parameter $\alpha_j = 0$,
$\mat W'$ would not have values only from $\pm 1$ but from $\{-1, 0, 1\}$.
This is not an issue as Clingo works by default above integers.
If 0-values were issue, they could be removed either by offsetting bias
parameter or by modifying the BNN structure to remove the constant node.
(This has not been implemented in my work.)

\begin{equation*}
    \vec \alpha_j = 0:\, \vec b'_j = \left\{
        \begin{matrix}
            \langle\vec 1, \vec 1\rangle + 1 & \mathrm{if}\,-\vec\gamma_j \geq 0;\\
            -\langle\vec 1, \vec 1\rangle - 1 & \mathrm{otherwise}
        \end{matrix}
    \right.
\end{equation*}

In my implementation I am also encoding inputs as binary values $\{0, 1\}$.

\begin{equation*}
    \begin{matrix}
        \vec x = 2 \vec x^{(b)} - \vec1\\
        \langle \vec x, \mat W'_{:, j}\rangle + \vec b'_j \geq 0\\
        \langle 2\vec x^{(b)} - \vec1, \mat W'_{:, j}\rangle + \vec b'_j \geq 0\\
        2\langle \vec x^{(b)}, \mat W'_{:, j}\rangle - \langle\vec1, \mat W'_{:,j}\rangle + \vec b'_j \geq 0\\
        \langle \vec x^{(b)}, \mat W'_{:, j}\rangle + \frac{- \langle\vec1, \mat W'_{:,j}\rangle + \vec b'_j}{2} \geq 0\\
    \end{matrix}
\end{equation*}

Still each block could be computed using one weight matrix and one bias vector.
Additionally, as $\langle\vec x^{(b)}, \mat'_{:, j}\rangle$ is an integer value,
equation is equivalent to

\begin{equation*}
    \langle \vec x^{(b)}, \mat W'_{:, j}\rangle + \left\lfloor \frac{- \langle\vec1, \mat W'_{:,j}\rangle + \vec b'_j}{2} \right\rfloor \geq 0\\
\end{equation*}

As for the output block, it has already just one weight matrix and one bias
vector. The implementation of ARGMAX layer is mainly Clingo-based.
The only needed transformation is to split bias to whole and decimal part.

\begin{equation*}
    \vec b = \vec b' + \vec p, \, \vec b'_j = \lfloor \vec b_j \rfloor
\end{equation*}

$\vec p$ is then used as main order of outputs.
% TODO: Try to order the outputs by this order? -> would remove one aggregate

Implementation of these transformations in Python is added as [appendix] % TODO

\section{Encoding of BNN in Clingo}

Implementation of BNN computation is avalible as [appendix] % TODO

\section{Encoding of input regions}

% TODO: To be implemented
% Definice input region
An input region of BNN is some subset of inputs on which the analysis is
performed. I have implemented both of input region types from [BNNQuanalys],
that is input region based on Hamming distance ($R_H$) and input region with
fixed indices ($R_I$).

The quantitative analysis of said BNN is a problem to say how many inputs


% - definice problému
% - proč je moje řešení korektní
% - kódování sítě, výpočtu na ní
% - kódování vstupu, výstupu
% - využití agregátních fcí - mohlo by vést ke zlepšení?
% 	+ agg funkce jako speciální případ OBDD

\chapter{Evaluation}

% - ohodnocení modelu
% 	+ je rozdíl mezi hledání, které vrátí málo výsledků vůči hodně výsledkům
% 	+ time to first / time to all
% 	+ clingo je (asi) optimalizované na hledání jednoho řešení, tady je chceme všechny
% 	+ -> Je rozumné hledat od nejvíce svázaného zadání a postupně rozšiřovat?
% - Jaký je rozumný limit pro velikost verifikované sítě
% 	+ je problém spíše s širokými/hlubokými sítěmi
% - jak si vede vůči původnímu zdroji, dalším implementacím
% - TODO: asi to vrazit na Metacentrum či tak něco? - můj ntb by pravděpodobně zkresloval výsledky

\chapter{Discussion}

% - má ASP pro další verifikaci význam?
% - je možnost, že by přinesl alespoň relevantní benchmark pro další verifikační algoritmy (např. nad obecnými NN?)
% 	+ není velká možnost, že by porazil algoritmy přizpůsobené přímo pro daný problém, ale je tu možnost že by byl fajn dokud nebudeme mít něco silnějšího než ASP pro problém

\end{document}
???END
